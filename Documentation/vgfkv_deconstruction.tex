
\documentclass{article}
%%%%%%%%%%%%%%%%%%%%%%%%%%%%%%%%%%%%%%%%%%%%%%%%%%%%%%%%%%%%%%%%%%%%%%%%%%%%%%%%%%%%%%%%%%%%%%%%%%%%%%%%%%%%%%%%%%%%%%%%%%%%%%%%%%%%%%%%%%%%%%%%%%%%%%%%%%%%%%%%%%%%%%%%%%%%%%%%%%%%%%%%%%%%%%%%%%%%%%%%%%%%%%%%%%%%%%%%%%%%%%%%%%%%%%%%%%%%%%%%%%%%%%%%%%%%
\usepackage{amssymb}

%TCIDATA{OutputFilter=LATEX.DLL}
%TCIDATA{Version=5.50.0.2960}
%TCIDATA{<META NAME="SaveForMode" CONTENT="1">}
%TCIDATA{BibliographyScheme=Manual}
%TCIDATA{Created=Wednesday, July 01, 2020 17:46:22}
%TCIDATA{LastRevised=Thursday, October 29, 2020 12:27:09}
%TCIDATA{<META NAME="GraphicsSave" CONTENT="32">}
%TCIDATA{<META NAME="DocumentShell" CONTENT="Scientific Notebook\Blank Document">}
%TCIDATA{Language=American English}
%TCIDATA{CSTFile=Math with theorems suppressed.cst}
%TCIDATA{PageSetup=72,72,72,72,0}
%TCIDATA{AllPages=
%F=36,\PARA{038<p type="texpara" tag="Body Text" >\hfill \thepage}
%}


\newtheorem{theorem}{Theorem}
\newtheorem{acknowledgement}[theorem]{Acknowledgement}
\newtheorem{algorithm}[theorem]{Algorithm}
\newtheorem{axiom}[theorem]{Axiom}
\newtheorem{case}[theorem]{Case}
\newtheorem{claim}[theorem]{Claim}
\newtheorem{conclusion}[theorem]{Conclusion}
\newtheorem{condition}[theorem]{Condition}
\newtheorem{conjecture}[theorem]{Conjecture}
\newtheorem{corollary}[theorem]{Corollary}
\newtheorem{criterion}[theorem]{Criterion}
\newtheorem{definition}[theorem]{Definition}
\newtheorem{example}[theorem]{Example}
\newtheorem{exercise}[theorem]{Exercise}
\newtheorem{lemma}[theorem]{Lemma}
\newtheorem{notation}[theorem]{Notation}
\newtheorem{problem}[theorem]{Problem}
\newtheorem{proposition}[theorem]{Proposition}
\newtheorem{remark}[theorem]{Remark}
\newtheorem{solution}[theorem]{Solution}
\newtheorem{summary}[theorem]{Summary}
\newenvironment{proof}[1][Proof]{\noindent\textbf{#1.} }{\ \rule{0.5em}{0.5em}}
\input{tcilatex}
\begin{document}


\section{VGFKV}

This code sets up a series of unit vectors in the direction of the outgoing
plane waves.

\section{Why we need this}

For VGFFMC We want to find the scattered field from a particle being hit by
a plane wave. The code sould find $\mathbf{E}$

In the dissertation we defined 
\begin{equation}
\mathbf{F}(\mathbf{r})\equiv \mathbf{E}(\mathbf{r})(1+\frac{4\pi }{3}\chi 
\mathbf{(r}))  \tag{38}  \label{EtoF}
\end{equation}%
We want the electric field so we want%
\[
\mathbf{E}(\mathbf{r})=\frac{\mathbf{F}(\mathbf{r})}{(1+\frac{4\pi }{3}\chi 
\mathbf{(r}))} 
\]

so we ned to calculate $\mathbf{F}(\mathbf{r})$ and $\chi \mathbf{(r})$

\begin{equation}
\tilde{F}_{\beta j}\mathbf{=}\sum_{N=1}^{N_{k}}a_{Nj}\Psi _{\beta N}
\label{CPSI}
\end{equation}

The $\Psi _{\beta N}$ are a series of plane waves%
\begin{equation}
\Psi _{\beta N}=e^{imk\mathbf{\hat{k}}_{N}\mathbf{\cdot r}_{\beta }}
\end{equation}%
and the $a_{Nj}$ is a set of coefficients. So our $E$ field comes out as a
series expansion where our explansion funcitons are $\Psi _{\beta N}$. We
need the $a_{Nj}$

The $a_{Nj}$ come from a matrix inversion 
\begin{equation}
\left( \mathbf{a}\right) =\left( \mathbf{H}\right) ^{-1}\left( \mathbf{Y}%
\right)  \label{aHY}
\end{equation}%
where $\mathbf{H}$ and $\mathbf{Y}$ are given by 
\begin{eqnarray}
H_{MlNj} &=&\Psi _{\gamma M}^{\ast }\left( 1_{\alpha i\gamma l}\mathbf{-}%
W_{\beta }^{\ast }G_{\alpha i\gamma l}^{\ast }\right) \left( 1_{\alpha
i\beta j}\mathbf{-}W_{\beta }G_{\alpha i\beta j}\right) \Psi _{\beta N} 
\nonumber \\
Y_{Nj} &=&\left( 1_{\alpha i\beta j}\mathbf{-}W_{\beta }G_{\alpha i\beta
j}\right) ^{\ast }E_{\alpha i}^{o}\Psi _{\beta N}^{\ast }
\end{eqnarray}

or more compactlyl (and how it is done in the code) 
\begin{equation}
T_{\alpha iNj}=\sum_{\beta }\left( 1_{\alpha i\beta j}\mathbf{-}G_{\alpha
i\beta j}W_{\beta }\right) \Psi _{\beta N}
\end{equation}%
Using $T_{\alpha iNj},$ $H_{MlNj}$ and $Y_{Nj}$ can be expressed as

\begin{eqnarray}
H_{MlNj} &=&\sum_{\alpha }\sum_{i}T_{\alpha iMl}^{\ast }T_{\alpha iNj}
\label{HTYT} \\
Y_{Nj} &=&\sum_{\alpha }\sum_{i}T_{\alpha iNj}^{\ast }E_{\alpha i}^{o} 
\nonumber
\end{eqnarray}

The $T_{\alpha iNj}$ depend on

\begin{equation}
\Psi _{\beta N}=e^{imk\mathbf{\hat{k}}_{N}\mathbf{\cdot r}_{\beta }}
\end{equation}%
which depend on the $\mathbf{\hat{k}}_{N}$

These are read in from a file. That file is created by vgfkv.f. And it looks
like it makes a $k_{N}$ in equally spaced increments in the $\theta $ and $%
\phi $ directions. \FRAME{ftbpF}{1.5238in}{1.4183in}{0pt}{}{}{Figure}{%
\special{language "Scientific Word";type "GRAPHIC";maintain-aspect-ratio
TRUE;display "USEDEF";valid_file "T";width 1.5238in;height 1.4183in;depth
0pt;original-width 3.9418in;original-height 3.6668in;cropleft "0";croptop
"1";cropright "1";cropbottom "0";tempfilename
'QIZ08401.wmf';tempfile-properties "XPR";}}So we have what looks like a set
of arrows in a spherical shape pointing away from the particle. Maybe we
could experiment with non-uniform distributions of these? 

\end{document}
