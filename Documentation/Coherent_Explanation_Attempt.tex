
\documentclass{article}
%%%%%%%%%%%%%%%%%%%%%%%%%%%%%%%%%%%%%%%%%%%%%%%%%%%%%%%%%%%%%%%%%%%%%%%%%%%%%%%%%%%%%%%%%%%%%%%%%%%%%%%%%%%%%%%%%%%%%%%%%%%%%%%%%%%%%%%%%%%%%%%%%%%%%%%%%%%%%%%%%%%%%%%%%%%%%%%%%%%%%%%%%%%%%%%%%%%%%%%%%%%%%%%%%%%%%%%%%%%%%%%%%%%%%%%%%%%%%%%%%%%%%%%%%%%%
\usepackage{amsfonts}
\usepackage{amsmath}

\setcounter{MaxMatrixCols}{10}
%TCIDATA{OutputFilter=LATEX.DLL}
%TCIDATA{Version=5.50.0.2960}
%TCIDATA{<META NAME="SaveForMode" CONTENT="1">}
%TCIDATA{BibliographyScheme=Manual}
%TCIDATA{Created=Friday, April 01, 2022 17:30:26}
%TCIDATA{LastRevised=Friday, April 01, 2022 19:27:38}
%TCIDATA{<META NAME="GraphicsSave" CONTENT="32">}
%TCIDATA{<META NAME="DocumentShell" CONTENT="Standard LaTeX\Standard LaTeX Article">}
%TCIDATA{Language=American English}
%TCIDATA{CSTFile=40 LaTeX article.cst}

\newtheorem{theorem}{Theorem}
\newtheorem{acknowledgement}[theorem]{Acknowledgement}
\newtheorem{algorithm}[theorem]{Algorithm}
\newtheorem{axiom}[theorem]{Axiom}
\newtheorem{case}[theorem]{Case}
\newtheorem{claim}[theorem]{Claim}
\newtheorem{conclusion}[theorem]{Conclusion}
\newtheorem{condition}[theorem]{Condition}
\newtheorem{conjecture}[theorem]{Conjecture}
\newtheorem{corollary}[theorem]{Corollary}
\newtheorem{criterion}[theorem]{Criterion}
\newtheorem{definition}[theorem]{Definition}
\newtheorem{example}[theorem]{Example}
\newtheorem{exercise}[theorem]{Exercise}
\newtheorem{lemma}[theorem]{Lemma}
\newtheorem{notation}[theorem]{Notation}
\newtheorem{problem}[theorem]{Problem}
\newtheorem{proposition}[theorem]{Proposition}
\newtheorem{remark}[theorem]{Remark}
\newtheorem{solution}[theorem]{Solution}
\newtheorem{summary}[theorem]{Summary}
\newenvironment{proof}[1][Proof]{\noindent\textbf{#1.} }{\ \rule{0.5em}{0.5em}}
\input{tcilatex}

\begin{document}

\title{Project Goal VGF Code}
\author{Todd Lines \\
%EndAName
BYU-I}
\maketitle

\section{What we want to do}

We want to calculate the average power scattered into a single direction
when light hits a small particle. Let's assume a plane wave incident on a
particle.

\FRAME{dtbpF}{3.7386in}{2.6221in}{0pt}{}{}{Figure}{\special{language
"Scientific Word";type "GRAPHIC";maintain-aspect-ratio TRUE;display
"USEDEF";valid_file "T";width 3.7386in;height 2.6221in;depth
0pt;original-width 4.6717in;original-height 3.2681in;cropleft "0";croptop
"1";cropright "1";cropbottom "0";tempfilename
'R9OSY50C.wmf';tempfile-properties "XPR";}}The light will scatter off of the
particle in different directions.\FRAME{dtbpF}{3.7075in}{2.8971in}{0in}{}{}{%
Figure}{\special{language "Scientific Word";type
"GRAPHIC";maintain-aspect-ratio TRUE;display "USEDEF";valid_file "T";width
3.7075in;height 2.8971in;depth 0in;original-width 3.6608in;original-height
2.853in;cropleft "0";croptop "1";cropright "1";cropbottom "0";tempfilename
'R9OSY50D.wmf';tempfile-properties "XPR";}}We want to get specific about
this and to pick a measure of how strong the scattered field is that is sent
off in a particular direction. Let's say we pick a plane that goes through
the particle and has the incident wave perpendicular to the plane. We could
pick a point on that plane and find the field or, because it is tradition,
we could find the average power scattered into that direction.

\FRAME{dtbpF}{5.7285in}{2.2701in}{0in}{}{}{Figure}{\special{language
"Scientific Word";type "GRAPHIC";maintain-aspect-ratio TRUE;display
"USEDEF";valid_file "T";width 5.7285in;height 2.2701in;depth
0in;original-width 5.6697in;original-height 2.2303in;cropleft "0";croptop
"1";cropright "1";cropbottom "0";tempfilename
'R9OSY50E.wmf';tempfile-properties "XPR";}}And if we do this in a circle
around the particle we get a good idea of how strong the field must be that
is scattered from the particle. And we could graph this average power
scattered into an angle. \FRAME{dtbpF}{5.1967in}{3.1583in}{0in}{}{}{Figure}{%
\special{language "Scientific Word";type "GRAPHIC";maintain-aspect-ratio
TRUE;display "USEDEF";valid_file "T";width 5.1967in;height 3.1583in;depth
0in;original-width 5.1413in;original-height 3.1133in;cropleft "0";croptop
"1";cropright "1";cropbottom "0";tempfilename
'R9OSY50G.wmf';tempfile-properties "XPR";}}where the value of the graph at $%
\phi =0\unit{%
%TCIMACRO{\U{b0}}%
%BeginExpansion
{{}^\circ}%
%EndExpansion
}$ is the power that scatters in the same direction of the incident wave
(the forward direction) and at $\phi =180\unit{%
%TCIMACRO{\U{b0}}%
%BeginExpansion
{{}^\circ}%
%EndExpansion
}$ is the amount of power that scatters backward (back scatter). \FRAME{dtbpF%
}{5.5564in}{2.9326in}{0pt}{}{}{Figure}{\special{language "Scientific
Word";type "GRAPHIC";maintain-aspect-ratio TRUE;display "USEDEF";valid_file
"T";width 5.5564in;height 2.9326in;depth 0pt;original-width
5.4993in;original-height 2.8885in;cropleft "0";croptop "1";cropright
"1";cropbottom "0";tempfilename 'R9OSY50F.wmf';tempfile-properties "XPR";}}%
In the example, there is a lot of forward scatter and a good amount of
backscatter, but very little scatter at $\phi =90\unit{%
%TCIMACRO{\U{b0}}%
%BeginExpansion
{{}^\circ}%
%EndExpansion
}.$

The power scattered into a specific direction is called the \emph{%
differential scattering cross section}. This is what we will calculate.

To do the calculation we need to solve for the fields. Power will be
proportional to the field squared. So we need the fields. The fields can be
calculated using a Green's function approach. 
\begin{equation}
\mathbf{E}(\mathbf{r})(1+\frac{4\pi }{3}\chi )=\mathbf{E}^{e}(\mathbf{r)+}%
\int d^{3}\mathbf{r}^{\prime }\mathbf{G}\left( \mathbf{r},\mathbf{r}^{\prime
}\right) \cdot \chi \mathbf{(r}^{\prime })\mathbf{E(r}^{\prime })
\label{E-field}
\end{equation}

where the Green's function comes from Poisson's equation of electrostatics%
\begin{equation*}
\nabla ^{2}G\left( \mathbf{r}\right) =\frac{\rho \left( \mathbf{r}\right) }{%
\varepsilon _{o}}
\end{equation*}%
This equation has solutions like 
\begin{equation*}
\nabla ^{2}G\left( \mathbf{r}\right) =\frac{1}{4\pi \varepsilon _{o}}\int 
\frac{\rho \left( \mathbf{r}^{\prime }\right) }{\left\vert \mathbf{r}-%
\mathbf{r}^{\prime }\right\vert }d\tau ^{\prime }
\end{equation*}%
integrated over the volume of the charge distribution. The Green's function
must solve the point source equation.   
\begin{equation*}
\nabla ^{2}G\left( \mathbf{r},\mathbf{r}^{\prime }\right) =-4\pi \delta
\left( \mathbf{r}-\mathbf{r}^{\prime }\right) 
\end{equation*}%
And in general it has a form like 
\begin{equation*}
G\left( \mathbf{r},\mathbf{r}^{\prime }\right) =\frac{1}{\left\vert \mathbf{r%
}-\mathbf{r}^{\prime }\right\vert }+\Omega \left( \mathbf{r},\mathbf{r}%
^{\prime }\right) 
\end{equation*}%
where $V$ is a function that satisfies 
\begin{equation*}
\nabla ^{2}\Omega \left( \mathbf{r},\mathbf{r}^{\prime }\right) =0
\end{equation*}%
Notice that we expect behavior kind of like $\frac{1}{\left\vert \mathbf{r}-%
\mathbf{r}^{\prime }\right\vert }$ because electric fields have an inverse
relationship to distance. But this is more complicated that just a field due
to a point charge.

And equation (\ref{E-field}) is hard to solve. To find a numeric solution we
will make some approximations. First off we will split the particle up into
pieces. Each piece we will approximate as a dipole scatterer because we know
how to calculate scattering from dipoles. So a small spherical particle
might look like this. \FRAME{dtbpF}{2.3782in}{2.0072in}{0in}{}{}{Figure}{%
\special{language "Scientific Word";type "GRAPHIC";maintain-aspect-ratio
TRUE;display "USEDEF";valid_file "T";width 2.3782in;height 2.0072in;depth
0in;original-width 2.3385in;original-height 1.9692in;cropleft "0";croptop
"1";cropright "1";cropbottom "0";tempfilename
'R9OSY50H.wmf';tempfile-properties "XPR";}}We need a method of defining the
locations and the strength of each dipole scatterer. We will call this
discretization the particle. Our method is to place the particle into a
square matrix. \FRAME{dtbpF}{5.1837in}{3.0908in}{0in}{}{}{Figure}{\special%
{language "Scientific Word";type "GRAPHIC";maintain-aspect-ratio
TRUE;display "USEDEF";valid_file "T";width 5.1837in;height 3.0908in;depth
0in;original-width 5.1283in;original-height 3.0467in;cropleft "0";croptop
"1";cropright "1";cropbottom "0";tempfilename
'R9OSY50I.wmf';tempfile-properties "XPR";}}If a matrix cell has part of the
particle in it, we will approximate it with a dipole scatterer. But we will
weight each dipole by how much of the particle is inside the cell. So the
cells in the corner of the matrix will be weighted less because there is
only a little of the particle in the cell. To determine how much of the
particle is in each cell, we divide up the cell into \textquotedblleft fine
structure\textquotedblright\ cells. And we count the number of fine
structure cells that have part of the particle in them. 

\FRAME{dtbpF}{3.1566in}{3.371in}{0in}{}{}{Figure}{\special{language
"Scientific Word";type "GRAPHIC";maintain-aspect-ratio TRUE;display
"USEDEF";valid_file "T";width 3.1566in;height 3.371in;depth
0in;original-width 3.1116in;original-height 3.3252in;cropleft "0";croptop
"1";cropright "1";cropbottom "0";tempfilename
'R9OSY50J.wmf';tempfile-properties "XPR";}}

Once we have a digitized description of the particle we need to find the
fields. But once again this it is hard to find the field in every direction.
So we will calculate the field as a series expansion of plane wave fields.
Each term will be a plane wave but each will have a different amplitude. And
we will take a finite set of terms. These terms will can be in different
directions. So we need to define at least a starting set of directions.
Because plane waves are of the form. 
\begin{equation*}
\Psi _{\beta N}=Ae^{imk\mathbf{\hat{k}}_{N}\mathbf{\cdot r}_{\beta }}
\end{equation*}%
where $N$ tells us we have the $Nth$ direction and where $\beta $ tells us
we are calculation the distance from the $\beta th$ dipole, we will refer to
the unit vectors in each of the expansion directions as $k$-vectors. \FRAME{%
dtbpF}{5.3177in}{1.8913in}{0pt}{}{}{Figure}{\special{language "Scientific
Word";type "GRAPHIC";maintain-aspect-ratio TRUE;display "USEDEF";valid_file
"T";width 5.3177in;height 1.8913in;depth 0pt;original-width
5.2624in;original-height 1.8533in;cropleft "0";croptop "1";cropright
"1";cropbottom "0";tempfilename 'R9OSY50K.wmf';tempfile-properties "XPR";}}

To calculate the fields from the dipole representation of the particle we
need a discredited field equation. We will assume a field with complex time
dependence.%
\begin{equation}
\mathbf{E}\left( \mathbf{r},t\right) =\mathbf{E(r})\exp (-i\omega t)+c.c.
\label{Field}
\end{equation}%
and as usual never write the time dependence. We want to solve%
\begin{equation}
\mathbf{E}(\mathbf{r})(1+\frac{4\pi }{3}\chi )=\mathbf{E}^{e}(\mathbf{r)+}%
\int d^{3}\mathbf{r}^{\prime }\mathbf{G}\left( \mathbf{r},\mathbf{r}^{\prime
}\right) \cdot \chi \mathbf{(r}^{\prime })\mathbf{E(r}^{\prime })
\end{equation}%
We define the term%
\begin{equation}
W\left( \mathbf{r}\right) =\frac{\chi \mathbf{(r})}{(1+\frac{4\pi }{3}\chi 
\mathbf{(r}))}  \label{W1}
\end{equation}%
And for convenience define a new field that has the $W\left( \mathbf{r}%
\right) $ as part of it because the $W\left( \mathbf{r}\right) $ is tedious
to write in all the equations.  
\begin{equation*}
\mathbf{F}(\mathbf{r})=\mathbf{E}(\mathbf{r})(1+\frac{4\pi }{3}\chi )
\end{equation*}%
We will call this new field the $F$-field. So our field equation is now 
\begin{equation}
\mathbf{F}(\mathbf{r})=\mathbf{E}^{e}(\mathbf{r)+}\int d^{3}\mathbf{r}%
^{\prime }\mathbf{G}\left( \mathbf{r},\mathbf{r}^{\prime }\right) \cdot
W\left( \mathbf{r}^{\prime }\right) \mathbf{F(r}^{\prime })  \label{F-field}
\end{equation}%
And this is what we need to discretize. Since our particle is made of
individual dipole scatterers, we can change the integral into a sum where we
add up the field contribution for each dipole. 
\begin{equation}
F_{\mu i}=E_{\mu i}^{in}\mathbf{+}\sum\limits_{\nu }G_{\mu i\nu j}W_{\nu
}F_{\nu j}  \label{AlgF}
\end{equation}%
where, for convenience, the cell dimension $d_{\mu }^{3}$ is included in the 
$W$ term. 
\begin{equation}
W_{\nu }=Wd_{\nu }^{3}
\end{equation}%
Symbolically, this may be written as 
\begin{equation}
\mathbf{F}=\mathbf{E}^{in}+\mathbf{GWF}
\end{equation}

If we knew the F-field and the incident field and the green's function we
could solve this. But if we knew all of this we would already have the
solution! But let's say we do know the fields, then we could write 
\begin{equation}
0=\mathbf{F-E}^{in}-\mathbf{GWF}
\end{equation}%
which seems obvious but unhelpful. But we want a numerical solution that is
a series expansion. So we can guess that the F-field is a series expansion
with plane waves as the basis of the expansion. 
\begin{equation}
\tilde{F}_{\beta j}\mathbf{=}\sum_{N=1}^{N_{k}}a_{Nj}\Psi _{\beta N}
\end{equation}%
But of course this isn't exactly right and we don't know the coefficients $%
a_{Nj}$ so $\mathbf{F-E}^{in}-\mathbf{GWF}$ won't be exactly zero. We could
define an error term $\mathbf{\varepsilon }$ such that 
\begin{equation}
\mathbf{\varepsilon }=\mathbf{F}-\mathbf{E}^{in}-\mathbf{GWF}
\end{equation}%
Writing components yields 
\begin{equation}
\varepsilon _{\alpha i}\mathbf{=}\sum_{\beta }\sum_{j}\left( 1_{\alpha
i\beta j}\mathbf{-}G_{\alpha i\beta j}W_{\beta }\right) F_{\beta j}\mathbf{-}%
E_{\alpha i}^{in}
\end{equation}%
Here the factor 
\begin{equation}
F_{\alpha i}^{st}=-W_{\alpha }G_{\alpha i\alpha j}F_{\alpha j}  \label{AST}
\end{equation}%
is the specially defined self term discussed below as the \textquotedblleft
self term.\textquotedblright\ Note that 
\begin{equation}
1_{\alpha i\beta j}=\delta _{\alpha \beta }\delta _{ij}
\end{equation}

So we have made some progress. If we knew the coefficients $a_{Nj}$ we could
find out how bad our approximation is to the fields by calculating $\mathbf{%
\varepsilon .}$ We want to make an attempt at this. 

Now, because the equations are getting long, define the quantity 
\begin{equation}
M_{\alpha i\beta j}=\left( 1_{\alpha i\beta j}\mathbf{-}W_{\beta }G_{\alpha
i\beta j}\right) 
\end{equation}%
where the $\alpha ,$ $i,$ $\beta ,$and $j$ are not summed over. Then 
\begin{equation}
\varepsilon _{\alpha i}\mathbf{=}\sum_{\beta }\sum_{j}\left( M_{\alpha
i\beta j}F_{\beta j}\mathbf{-}E_{\alpha i}^{o}\right)   \label{M}
\end{equation}%
which looks shorter.

But thinking of our notation for fields, the quantity $\varepsilon _{\alpha
i}$ has complex components and is not a single value. It would be easier if
we had a real number that told us how big our error in calculating the
fields would be. So let's define a new function  
\begin{equation*}
\Phi =\sum_{\alpha }\sum_{i}\mathbf{\varepsilon }_{\alpha i}\mathbf{%
\varepsilon }_{\alpha i}^{\ast }
\end{equation*}%
which would be real because of the conjugate multiply and would be a single
number because of the sums. We want to minimize $\Phi .$

\begin{equation}
\Phi =\sum_{\alpha }\sum_{i}\left( \sum_{\beta }\sum_{j}\left( M_{\alpha
i\beta j}F_{\beta j}\mathbf{-}E_{\alpha i}^{o}\right) \right) \left(
\sum_{\gamma }\sum_{l}\left( M_{\alpha i\gamma l}F_{\gamma \lambda }\mathbf{-%
}E_{\alpha i}^{o}\right) \right) ^{\ast }
\end{equation}%
or

\begin{equation}
\Phi =\sum_{\alpha }\sum_{i}\mathbf{\varepsilon }_{\alpha i}\mathbf{%
\varepsilon }_{\alpha i}^{\ast }=\sum_{\alpha }\sum_{i}\sum_{\beta
}\sum_{j}\sum_{\gamma }\sum_{l}\left( F_{\beta j}M_{\beta j\alpha
i}^{T}M_{\alpha i\gamma l}^{\ast }F_{\gamma l}^{\ast }-E_{\alpha i}^{o\ast
}M_{\alpha i\beta j}F_{\beta j}-E_{\alpha i}^{o}M_{\alpha i\beta j}^{\ast
}F_{\beta j}^{\ast }+E_{\alpha i}^{o}E_{\alpha i}^{o\ast }\right) 
\label{PHI}
\end{equation}%
Which is getting long again. Again for convenience two additional quantities
are defined which will shorten the notation 
\begin{eqnarray}
K_{\gamma l\beta j} &=&M_{\gamma l\alpha i}^{\ast T}M_{\alpha i\beta j} \\
X_{\beta j}^{\ast } &=&E_{\alpha i}^{o\ast }M_{\alpha i\beta j}
\end{eqnarray}%
Then equation (\ref{PHI}) can be written compactly as 
\begin{equation}
\Phi =\sum_{\alpha }\sum_{i}\sum_{\beta }\sum_{j}\sum_{\gamma
}\sum_{l}\left( F_{\gamma l}^{\ast }K_{\gamma l\beta j}F_{\beta j}-\left[
X_{\beta j}^{\ast }F_{\beta j}+X_{\beta j}F_{\beta j}^{\ast }\right]
+E_{\alpha i}^{o}E_{\alpha i}^{o\ast }\right) 
\end{equation}%
And now we can put in our guess for $F.$ 
\begin{equation}
\tilde{F}_{\beta j}\mathbf{=}\sum_{N=1}^{N_{k}}a_{Nj}\Psi _{\beta N}
\end{equation}
The function $\mathbf{\tilde{F}}$ is substituted for $\mathbf{F.}$ The
functions $\Psi _{\beta N}$ we will assume to be of the form

\begin{equation}
\Psi _{\beta N}=e^{imk\mathbf{\hat{k}}_{N}\mathbf{\cdot r}_{\beta }}
\end{equation}%
which depend on the $\mathbf{\hat{k}}_{N}$

Then equation (\ref{PHI}) is approximately given by 
\begin{equation}
\Phi =\sum_{\alpha }\sum_{i}\sum_{\beta }\sum_{j}\sum_{\gamma
}\sum_{l}\sum_{N=1}^{N_{k}}\sum_{M=1}^{N_{k}}\left( \Psi _{\gamma M}^{\ast
}a_{Ml}^{\ast }K_{\gamma l\beta j}a_{Nj}\Psi _{\beta N}-\left[ X_{\beta
j}^{\ast }\Psi _{\beta N}a_{Nj}+X_{\beta j}\Psi _{\beta N}^{\ast
}a_{Nj}^{\ast }\right] +E_{\alpha i}^{o}E_{\alpha i}^{o\ast }\right) 
\end{equation}%
Now the notation can again be simplified by defining two functions 
\begin{eqnarray}
H_{MlNj} &=&\sum_{\gamma }\sum_{\beta }\Psi _{\gamma M}^{\ast }K_{\gamma
l\beta j}\Psi _{\beta N} \\
Y_{Nj}^{\ast } &=&\sum_{\beta }\Psi _{N\beta }^{T}X_{\beta j}^{\ast }
\end{eqnarray}%
Then the final form of equation (\ref{PHI}) is 
\begin{equation}
\Phi =\sum_{\alpha
}\sum_{i}\sum_{j}\sum_{l}\sum_{N=1}^{N_{k}}\sum_{M=1}^{N_{k}}a_{Ml}^{\ast
}H_{MlNj}a_{Nj}-\left[ Y_{Nj}^{\ast }a_{Nj}+Y_{Nj}a_{Nj}^{\ast }\right]
+E_{\alpha i}^{o}E_{\alpha i}^{o\ast }  \label{PhiSimp}
\end{equation}

But we don't know the expansion coefficients $a_{Nj}$ and $a_{Ml}^{\ast }.$
We need to find these. We can use a variational technique. We can find a set
of $a_{Nj}$ and $a_{Ml}^{\ast }$ that make $\Phi $ a minimum. An d we know
how to find a minimum in an equation, we take a derivative and set it equal
to zero. 

Let's differentiate $\Phi $ with respect to $a_{Li}^{\ast },$%
\begin{equation*}
\frac{\partial \Phi }{\partial a_{Li}^{\ast }}=\sum_{\alpha
}\sum_{i}\sum_{j}\sum_{l}\sum_{N=1}^{N_{k}}\sum_{M=1}^{N_{k}}a_{Ml}^{\ast
}H_{MlNj}a_{Nj}-\left[ Y_{Nj}^{\ast }a_{Nj}+Y_{Nj}a_{Nj}^{\ast }\right]
+E_{\alpha i}^{o}E_{\alpha i}^{o\ast }
\end{equation*}%
most terms drop out because $a_{Li}^{\ast }$ is just one coefficient, so    
\begin{equation}
\frac{\partial \Phi }{\partial a_{Li}^{\ast }}=\sum_{N=1}^{N_{k}}\sum_{j}%
\left( H_{LiNj}a_{Nj}-Y_{Li}\right)   \label{DerPhi}
\end{equation}%
and if we set this equal to zero we get%
\begin{equation}
0=\sum_{N=1}^{N_{k}}\sum_{j}\left( H_{LiNj}a_{Nj}-Y_{Li}\right) 
\end{equation}%
\begin{equation}
\sum_{N=1}^{N_{k}}\sum_{j}H_{LiNj}a_{Nj}=Y_{Li}
\end{equation}

can be written in matrix form as 
\begin{equation}
\left( \mathbf{H}\right) \left( \mathbf{a}\right) =\left( \mathbf{Y}\right) 
\label{HaY}
\end{equation}%
The solution for $\left( \mathbf{a}\right) $ is given by 
\begin{equation}
\left( \mathbf{a}\right) =\left( \mathbf{H}\right) ^{-1}\left( \mathbf{Y}%
\right)   \label{aHY1}
\end{equation}

We can use a matrix solver to find a solution for the $\left( \mathbf{a}%
\right) $ values. Then we can put them back ion  
\begin{equation}
\tilde{F}_{\beta j}\mathbf{=}\sum_{N=1}^{N_{k}}a_{Nj}e^{imk\mathbf{\hat{k}}%
_{N}\mathbf{\cdot r}_{\beta }}  \label{Ftilde}
\end{equation}%
to find the fields.  The $\mathbf{\hat{k}}_{N}$ or \textquotedblleft $k$%
-vectors\textquotedblright\ must be chosen to represent the internal wave
well. A \textquotedblleft goodness of choice\textquotedblright\ criterion
for the number and direction of the $\mathbf{\hat{k}}_{N}$ vectors results
from evaluating $\Phi $ explicitly. Observing the values of the $a_{Nj}$ can
determine which of the $k$-vectors are not contributing well. If the
magnitude of an $a_{N}$ is small, then the $\mathbf{\hat{k}}_{N}$ is not
well placed or may not be needed.

For the sake of computation, a temporary variable is employed in the
calculations that follow. 
\begin{equation}
T_{\alpha iNj}=\sum_{\beta }\left( 1_{\alpha i\beta j}\mathbf{-}G_{\alpha
i\beta j}W_{\beta }\right) \Psi _{\beta N}
\end{equation}
Using $T_{\alpha iNj},$ $H_{MlNj}$ and $Y_{Nj}$ can be expressed as

\begin{eqnarray}
H_{MlNj} &=&\sum_{\alpha }\sum_{i}T_{\alpha iMl}^{\ast }T_{\alpha iNj} \\
Y_{Nj} &=&\sum_{\alpha }\sum_{i}T_{\alpha iNj}^{\ast }E_{\alpha i}^{o} 
\notag
\end{eqnarray}%
This is the form of the quantities used in the actual algorithms.

Then since we found the $a_{Nj}$ with the matrix inversion we can calculate
the 
\begin{equation}
\tilde{F}_{\beta j}\mathbf{=}\sum_{N=1}^{N_{k}}a_{Nj}e^{imk\mathbf{\hat{k}}%
_{N}\mathbf{\cdot r}_{\beta }}
\end{equation}%
and these are related to the electric field like this

\begin{equation*}
\mathbf{F}(\mathbf{r})=\mathbf{E}(\mathbf{r})(1+\frac{4\pi }{3}\chi ) 
\end{equation*}%
so 
\begin{equation*}
\frac{\mathbf{F}(\mathbf{r})}{(1+\frac{4\pi }{3}\chi )}=\mathbf{E}(\mathbf{r}%
) 
\end{equation*}%
and we have our electric field.

This is what the code is supposed to do.

\end{document}
