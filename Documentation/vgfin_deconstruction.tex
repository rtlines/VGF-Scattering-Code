
\documentclass{article}
%%%%%%%%%%%%%%%%%%%%%%%%%%%%%%%%%%%%%%%%%%%%%%%%%%%%%%%%%%%%%%%%%%%%%%%%%%%%%%%%%%%%%%%%%%%%%%%%%%%%%%%%%%%%%%%%%%%%%%%%%%%%%%%%%%%%%%%%%%%%%%%%%%%%%%%%%%%%%%%%%%%%%%%%%%%%%%%%%%%%%%%%%%%%%%%%%%%%%%%%%%%%%%%%%%%%%%%%%%%%%%%%%%%%%%%%%%%%%%%%%%%%%%%%%%%%
%TCIDATA{OutputFilter=LATEX.DLL}
%TCIDATA{Version=5.50.0.2960}
%TCIDATA{<META NAME="SaveForMode" CONTENT="1">}
%TCIDATA{BibliographyScheme=Manual}
%TCIDATA{Created=Thursday, June 25, 2020 11:45:21}
%TCIDATA{LastRevised=Friday, February 11, 2022 10:50:56}
%TCIDATA{<META NAME="GraphicsSave" CONTENT="32">}
%TCIDATA{<META NAME="DocumentShell" CONTENT="Scientific Notebook\Blank Document">}
%TCIDATA{Language=American English}
%TCIDATA{CSTFile=Math with theorems suppressed.cst}
%TCIDATA{PageSetup=72,72,72,72,0}
%TCIDATA{AllPages=
%F=36,\PARA{038<p type="texpara" tag="Body Text" >\hfill \thepage}
%}


\newtheorem{theorem}{Theorem}
\newtheorem{acknowledgement}[theorem]{Acknowledgement}
\newtheorem{algorithm}[theorem]{Algorithm}
\newtheorem{axiom}[theorem]{Axiom}
\newtheorem{case}[theorem]{Case}
\newtheorem{claim}[theorem]{Claim}
\newtheorem{conclusion}[theorem]{Conclusion}
\newtheorem{condition}[theorem]{Condition}
\newtheorem{conjecture}[theorem]{Conjecture}
\newtheorem{corollary}[theorem]{Corollary}
\newtheorem{criterion}[theorem]{Criterion}
\newtheorem{definition}[theorem]{Definition}
\newtheorem{example}[theorem]{Example}
\newtheorem{exercise}[theorem]{Exercise}
\newtheorem{lemma}[theorem]{Lemma}
\newtheorem{notation}[theorem]{Notation}
\newtheorem{problem}[theorem]{Problem}
\newtheorem{proposition}[theorem]{Proposition}
\newtheorem{remark}[theorem]{Remark}
\newtheorem{solution}[theorem]{Solution}
\newtheorem{summary}[theorem]{Summary}
\newenvironment{proof}[1][Proof]{\noindent\textbf{#1.} }{\ \rule{0.5em}{0.5em}}
\input{tcilatex}
\begin{document}


\section{ VGFIN}

\subsection{Original description:}

Program to set up a particle dipole array for the VGF code. A particle
(sphere, oblate or prolate spheroid) is split into a cubic array of dipole
cells (cubic cells with a dipole at the center). A course array of NSID%
\symbol{94}3 dipole cells is defined using a cubic lattice. The particle is,
in effect, carved out of a NSID\symbol{94}3 cube of dipole cells. The course
array cells are divided into NLSID fine array cells. Each of the fine array
cells is tested to see if it is within the particle geometry. If so, it is
counted. The course array cell is weighted in volume (buy adjusting the cell
length on a side, (D) according to how many of it's fine array cells are in
the particle.

\subsection{Units}

Angles are input in degrees and converted to radians. Lengths are all
relative, that is, if you input a wavelength of 10 um then all other lengths
must be in um. You may use m or cm or furlongs if you wish as long as all
lengths are in the same units.

\subsection{Time Dependence}

The VGF code uses the time dependence exp(-iwt). This code (VGFIN) receives
the index of refraction from the user and modifies the sign of the imaginary
part to fit the VGF time dependence convention (i.e. the imaginary part
should be positive).

\subsection{Geometry}

The particle is aligned with it's symmetry axis along the Z-axis. The
incident plane wave may be rotated

around the particle by the angles alpha, beta, and gamma (corresponding to
the Euler angles as given in Arfkin, Mathematical Methods for Physicists,
2nd Ed., Academic Press, NY, 1970). Gamma is effectively the polarization
angle around the incident (k-hat) direction and is input as such.

\subsection{Variable Definitions}

\subsubsection{Initialized variables}

The code initializes variables in lines 46-66. ForTran requires variables to
be declared like in C++. It uses parenthesis to indicate an array.

We start with 
\[
\begin{tabular}{|l|l|}
\hline
the maximum number of possible dipole cells & $NMAX$ \\ \hline
\end{tabular}%
\]

\[
\begin{tabular}{|l|l|}
\hline
wavelength & $W$ \\ \hline
number of dipoles in along the major axis & $NSID$ \\ \hline
fine structure divisions per dipole cell side & $NLSID$ \\ \hline
Particle symmetry semi-axis & $RAD$ \\ \hline
Incident direction & $\left( \alpha ,\beta ,\gamma \right) $ \\ \hline
polarization angle & $\psi $ \\ \hline
real and imaginary parts of index of refraction & $MR,$ $MI$ \\ \hline
index of refraction (complex) & $M$ \\ \hline
$\pi $ & $PI$ \\ \hline
Aspect ratio, ration of major to minor axis & $AR$ \\ \hline
\end{tabular}%
\]%
\[
\begin{tabular}{|l|l|}
\hline
skin depth & $SD$ \\ \hline
dipole course array cell length on a side & $TD$ \\ \hline
number of cell centers on a side & $DRF$ \\ \hline
Fine Array cell size $d$ & $LD$ \\ \hline
Inverse of the fine structure cell volume & $FRAC$ \\ \hline
Number of Cells $N^{3}$ & $NCUB$ \\ \hline
\end{tabular}%
\]%
\[
\begin{tabular}{|l|l|}
\hline
number of cells cubed & $NCUB$ \\ \hline
number of fine structure cells that are inside the particle & $INCNT$ \\ 
\hline
Number of dipoles that we actually use & $NUSE$ \\ \hline
Dipole weighting factors & $D(NMAX)$ \\ \hline
Temporary variable, position of fine structure cell & $RF\left( 3\right) $
\\ \hline
Locations of dipoles, $x,$ $y,$ and $z$ components. & $R(NMAX,3)$ \\ \hline
$i=\sqrt{-1}$ & $CI$ \\ \hline
permittivity & $EPS$ \\ \hline
real and imaginary part of the permittivity & $ER,$ $EI$ \\ \hline
\end{tabular}%
\]

\subsubsection{Calculated variables}

The skin depth is 
\[
SD=\frac{W}{2\pi MI} 
\]%
so long as $MI$ is not equal to zero. If not, then $SD=0$

This comes from assuming an exponential fall off for the plane wave as it
enters the dielectric material of the particle 
\[
E\left( x\right) =E_{o}e^{-\alpha x} 
\]%
the skin depth is the distance equal to the reciprocal of the attenuation
coefficient%
\[
x=\frac{1}{\alpha } 
\]%
so that 
\[
E\left( x\right) =E_{o}e^{-1} 
\]%
at this depth. For us 
\[
\alpha =kn_{I}=\frac{2\pi }{\lambda }n_{I}=\frac{2\pi }{W}M_{I} 
\]%
so the skin depth is 
\[
SD=\frac{W}{2\pi MI} 
\]%
This is interesting, but I\ don't think it is actually used.

\subsubsection{Dipole course array cell length}

The code is supposed to make a three dimensional array of dipoles in the
shape of the particle. We will split the volume of space into which we put
our particle into cubical dipole cells, then figure out which of the dipole
cells are actually part of the particle. \FRAME{dtbpF}{4.9805in}{2.9689in}{%
0in}{}{}{Figure}{\special{language "Scientific Word";type
"GRAPHIC";maintain-aspect-ratio TRUE;display "USEDEF";valid_file "T";width
4.9805in;height 2.9689in;depth 0in;original-width 4.9251in;original-height
2.9257in;cropleft "0";croptop "1";cropright "1";cropbottom "0";tempfilename
'QCR8CN02.wmf';tempfile-properties "XPR";}}

The dipole course array cell length on a side is how big the individual
dipole cells are.\FRAME{dtbpF}{2.4993in}{1.6535in}{0pt}{}{}{Figure}{\special%
{language "Scientific Word";type "GRAPHIC";maintain-aspect-ratio
TRUE;display "USEDEF";valid_file "T";width 2.4993in;height 1.6535in;depth
0pt;original-width 2.4587in;original-height 1.6172in;cropleft "0";croptop
"1";cropright "1";cropbottom "0";tempfilename
'QCPNLD0E.wmf';tempfile-properties "XPR";}} They are cubical, so the cell
volume is defined by the cell length.

\[
TD=\frac{2.0\max \left( RAD,\frac{RAD}{AR}\right) }{NSID} 
\]

We are going to need to find the center of each of these cells. We will
calculate this next bit over and over, so to save some run time, let's
precalculate it. More on what it means later.

\[
DRF=\frac{1}{2}\left( NSID-1\right) 
\]

Now we are going to break up each dipole cell into smaller cells to weight
the edge cells by how much of the dipole cell is in the actual particle. To
do this we make smaller little cubes inside the dipole cell that have side
lengths of $d$%
\[
d=\frac{TD}{NLSID}=LD 
\]%
We will call this the \textquotedblleft fine structure\textquotedblright 
\FRAME{dtbpF}{3.8821in}{2.7925in}{0pt}{}{}{Figure}{\special{language
"Scientific Word";type "GRAPHIC";maintain-aspect-ratio TRUE;display
"USEDEF";valid_file "T";width 3.8821in;height 2.7925in;depth
0pt;original-width 3.8337in;original-height 2.7501in;cropleft "0";croptop
"1";cropright "1";cropbottom "0";tempfilename
'QCPOVS0F.wmf';tempfile-properties "XPR";}}

\subsubsection{Inverse of the fine structure cell volume}

So I\ precalculated this.

\[
FRAC=\frac{1}{NLSID^{3}} 
\]%
because we will use it later.

\subsubsection{Loop to crate particle}

\FRAME{dtbpF}{2.4578in}{2.6757in}{0in}{}{}{Figure}{\special{language
"Scientific Word";type "GRAPHIC";maintain-aspect-ratio TRUE;display
"USEDEF";valid_file "T";width 2.4578in;height 2.6757in;depth
0in;original-width 2.4163in;original-height 2.6333in;cropleft "0";croptop
"1";cropright "1";cropbottom "0";tempfilename
'QCPQQN0G.wmf';tempfile-properties "XPR";}}

The loop is supposed to iterate over each dimension making dipole cell
locations. Each dipole cell center location is calculated as 
\[
x_{i}=\left( i-DRF\right) TD 
\]%
Lets try this,. Suppose 
\[
NSID=4 
\]%
then 
\begin{eqnarray*}
DRF &=&\frac{1}{2}\left( NSID-1\right) \\
&=&\frac{1}{2}\left( 4-1\right) \\
&=&\allowbreak \frac{3}{2}
\end{eqnarray*}%
then the cell locations in the $x$ direction would be, calculated like this%
\[
x_{i}=\left( i-DRF\right) TD 
\]%
and the first one would be 
\[
x_{0}=\left( 0-\frac{3}{2}\right) TD=-\frac{3}{2}TD 
\]%
and here are the rest%
\[
\begin{tabular}{|c|c|}
\hline
$i$ & $x\left( TD\right) $ \\ \hline
$0$ & $-\frac{3}{2}$ \\ \hline
$1$ & $-\frac{1}{2}$ \\ \hline
$2$ & $\frac{1}{2}$ \\ \hline
$3$ & $\frac{3}{2}$ \\ \hline
\end{tabular}%
\]

We do this for the $y$ and $z$ directions too.

For the $NSID=4$ case these would be the dipole cell center locations in the 
$x$ and $y$ directions (green dots in the next figure)

\FRAME{dtbpF}{5.4803in}{3.2629in}{0pt}{}{}{Figure}{\special{language
"Scientific Word";type "GRAPHIC";maintain-aspect-ratio TRUE;display
"USEDEF";valid_file "T";width 5.4803in;height 3.2629in;depth
0pt;original-width 6.6668in;original-height 3.9583in;cropleft "0";croptop
"1";cropright "1";cropbottom "0";tempfilename
'QCPR570H.wmf';tempfile-properties "XPR";}}

\subsubsection{Fine Structure}

Now in the same loop we are going to look at the fine structure boxes to see
how many of them are filled with the particle. The fine structure box
locations are given by

\[
RF(1)=x+LD\ast (0.5\ast (1.0-NLSID)+IX) 
\]%
for the $x$ direction and similarly for the $y$ and $z$ directions. The
quantity $IX$ is the fine structure loop counter. And what we do is to work
from the middle of the cube again. Let's take NLSID\ also equal to $4$%
\[
NLSID=4 
\]%
then%
\[
d=\frac{TD}{NLSID}=LD 
\]%
(LD\ stands for \textquotedblleft little d\textquotedblright ) In our case, 
\[
d=\frac{TD}{4}=\frac{1}{4}TD 
\]%
gives the side length of the fine structure box. So for our choices%
\[
RF(1)=x+LD\ast (0.5\ast (1.0-NLSID)+IX) 
\]%
becomes 
\[
RF(1)=x+\frac{1}{4}TD\ast (0.5\ast (1.0-4)+IX) 
\]
then if we start with $IX=0,$ we have 
\[
RF(1)=x+\frac{1}{4}TD\ast (0.5\ast (1.0-4)+0) 
\]
which would be 
\[
RF(1)=x+\frac{1}{4}TD\ast (-\frac{3}{2})) 
\]%
For $IX=1$%
\[
RF(1)=x+\frac{1}{4}TD\ast (0.5\ast (1.0-4)+1) 
\]%
\[
RF(1)=x+\frac{1}{4}TD\ast (0.5\ast (-3)+1) 
\]%
\[
RF(1)=x+\frac{1}{4}TD\ast ((-\frac{3}{2})+1) 
\]%
\[
RF(1)=x+\frac{1}{4}TD\ast ((-\frac{1}{2}) 
\]

and if $IX=2$ 
\[
RF(1)=x+\frac{1}{4}TD\ast (0.5\ast (1.0-4)+2) 
\]%
\[
RF(1)=x+\frac{1}{4}TD\ast ((-\frac{3}{2})+2) 
\]%
\[
RF(1)=x+\frac{1}{4}TD\ast ((\frac{1}{2}) 
\]%
and if $IX=3$%
\[
RF(1)=x+\frac{1}{4}TD\ast (0.5\ast (1.0-4)+3) 
\]%
\[
RF(1)=x+\frac{1}{4}TD\ast ((\frac{3}{2}) 
\]

and now let's think about the $x$ values. 
\[
\begin{tabular}{|c|c|}
\hline
$i$ & $x\left( TD\right) $ \\ \hline
$0$ & $-\frac{3}{2}$ \\ \hline
$1$ & $-\frac{1}{2}$ \\ \hline
$2$ & $\frac{1}{2}$ \\ \hline
$3$ & $\frac{3}{2}$ \\ \hline
\end{tabular}%
\]

Taking the first for $i=0$%
\begin{eqnarray*}
RF(1) &=&-\frac{3}{2}TD+\frac{1}{4}TD\ast (-\frac{3}{2}) \\
&=&-\frac{3}{2}TD+-\frac{3}{8}TD \\
&=&-\frac{15}{8}TD
\end{eqnarray*}
\[
RF(1)=-\frac{3}{2}TD+\frac{1}{4}TD\ast (-\frac{1}{2})=-\frac{13}{8}TD 
\]
\[
RF(1)=-\frac{3}{2}TD+\frac{1}{4}TD\ast (\frac{1}{2})=-\frac{11}{8}TD 
\]
\[
RF(1)=-\frac{3}{2}TD+\frac{1}{4}TD\ast (\frac{3}{2})=-\frac{9}{8}TD 
\]%
so the fine structure cell locations would look like this in an end view of
our particle space.\FRAME{dtbpF}{3.5379in}{3.2872in}{0in}{}{}{Figure}{%
\special{language "Scientific Word";type "GRAPHIC";maintain-aspect-ratio
TRUE;display "USEDEF";valid_file "T";width 3.5379in;height 3.2872in;depth
0in;original-width 3.4912in;original-height 3.2413in;cropleft "0";croptop
"1";cropright "1";cropbottom "0";tempfilename
'QCR4SW00.wmf';tempfile-properties "XPR";}}

and for $i=1$

\[
RF(1)=-\frac{1}{2}TD+\frac{1}{4}TD\ast (-\frac{3}{2})=\allowbreak -\frac{7}{8%
}TD 
\]%
\[
RF(1)=-\frac{1}{2}TD+\frac{1}{4}TD\ast (-\frac{1}{2})=-\frac{5}{8}TD 
\]
\[
RF(1)=-\frac{1}{2}TD+\frac{1}{4}TD\ast (\frac{1}{2})=-\frac{3}{8}TD 
\]
\[
RF(1)=-\frac{1}{2}TD+\frac{1}{4}TD\ast (\frac{3}{2})=-\frac{1}{8}TD 
\]

which would look like this\FRAME{dtbpF}{3.4048in}{3.5466in}{0in}{}{}{Figure}{%
\special{language "Scientific Word";type "GRAPHIC";maintain-aspect-ratio
TRUE;display "USEDEF";valid_file "T";width 3.4048in;height 3.5466in;depth
0in;original-width 3.3589in;original-height 3.4999in;cropleft "0";croptop
"1";cropright "1";cropbottom "0";tempfilename
'QCR4XZ01.wmf';tempfile-properties "XPR";}}

Note that we are going to reuse $RF(1),$ $RF(2),$ and $RF(3)$ so the code
doesn't put them into an array for every box. We just need to calculate them
to see if that location is inside our particle. Then we resuse the $RF$
variable for the next box.

\subsubsection{See if a fine structure cell is inside the particle}

The whole point of the fine structure is to see how much of the dipole cell
is inside the particle. We need to adjust for partially filled dipole cells.
Otherwise, all particles would be box like shapes. To do this adjusting we
use the function ISINSIDE. This function takes the spheroidal equation%
\[
1=\sqrt{\frac{x^{2}}{b^{2}}+\frac{y^{2}}{b^{2}}+\frac{z^{2}}{a^{2}}} 
\]%
where%
\[
b=\frac{a}{a_{r}} 
\]%
which defines the outer surface of our ellipsoidal particle. So if 
\[
1>\sqrt{\frac{x^{2}}{b^{2}}+\frac{y^{2}}{b^{2}}+\frac{z^{2}}{a^{2}}} 
\]%
we are inside the particle and if 
\[
1<\sqrt{\frac{x^{2}}{b^{2}}+\frac{y^{2}}{b^{2}}+\frac{z^{2}}{a^{2}}} 
\]%
we are out side the particle. In the code our ellipsoidal equation is given
as 
\[
1>\sqrt{\frac{RF\left( 1\right) }{\frac{RAD^{2}}{AR^{2}}}+\sqrt{\frac{%
RF\left( 2\right) }{\frac{RAD^{2}}{AR^{2}}}+\sqrt{\frac{RF\left( 2\right) }{%
RAD^{2}}+}}} 
\]%
where 
\[
RAD=a 
\]%
the semimajor axis and 
\[
\frac{RAD}{AR}=b 
\]%
so that 
\[
b=\frac{a}{a_{r}}=\frac{RAD}{AR} 
\]

If the fine structure location is inside the particle we count it with the
counter INCNT. And we do this for all the fine structure locations for the
dipole cell. Then we weight the dipole by the number of fine structure cells
that are inside the particle.

\subsubsection{Dipole weighting}

The strategy used to do this is to take the number of fine structure cells
that are inside, and weight the size of the dipole based on how many parts
of the dipole cell are filled. The scaling factor is given by 
\[
D=TD\left( \frac{INCNT}{_{NLSID^{3}}}\right) ^{\frac{1}{3}}
\]%
That is, take the number of fine structure cells that are in the particle
and divide by the total number of fine structure cells, and take the cubed
root. We have a relative volume, and have converted it into a relative
length. Multiply this by $TD$, the dipole cell size, to get an absolute
scaled size for the dipole.

\subsection{Particle representation visualization }

Th particle looks like an array of dipole spheres that are weighted in size
by how much of their volume contains part of the particle. \FRAME{dtbpF}{%
1.7469in}{1.7115in}{0pt}{}{}{Figure}{\special{language "Scientific
Word";type "GRAPHIC";maintain-aspect-ratio TRUE;display "USEDEF";valid_file
"T";width 1.7469in;height 1.7115in;depth 0pt;original-width
4.5921in;original-height 4.4996in;cropleft "0";croptop "1";cropright
"1";cropbottom "0";tempfilename
'particle_visualization.wmf';tempfile-properties "XNPR";}}

\subsection{Print to a file}

The program then prints everything to a file. The program uses old fashioned
ForTran formatted output. C++ has formatted output. The print statements
specify the exact number of digits in each output value. Line $100$ is the
format. Other write statements are more loose, letting the compiler use the
default number of digits.

\end{document}
