
\documentclass{article}
%%%%%%%%%%%%%%%%%%%%%%%%%%%%%%%%%%%%%%%%%%%%%%%%%%%%%%%%%%%%%%%%%%%%%%%%%%%%%%%%%%%%%%%%%%%%%%%%%%%%%%%%%%%%%%%%%%%%%%%%%%%%%%%%%%%%%%%%%%%%%%%%%%%%%%%%%%%%%%%%%%%%%%%%%%%%%%%%%%%%%%%%%%%%%%%%%%%%%%%%%%%%%%%%%%%%%%%%%%%%%%%%%%%%%%%%%%%%%%%%%%%%%%%%%%%%
\usepackage{amssymb}

%TCIDATA{OutputFilter=LATEX.DLL}
%TCIDATA{Version=5.50.0.2960}
%TCIDATA{<META NAME="SaveForMode" CONTENT="1">}
%TCIDATA{BibliographyScheme=Manual}
%TCIDATA{Created=Wednesday, July 01, 2020 17:46:22}
%TCIDATA{LastRevised=Monday, June 07, 2021 13:24:47}
%TCIDATA{<META NAME="GraphicsSave" CONTENT="32">}
%TCIDATA{<META NAME="DocumentShell" CONTENT="Scientific Notebook\Blank Document">}
%TCIDATA{Language=American English}
%TCIDATA{CSTFile=Math with theorems suppressed.cst}
%TCIDATA{PageSetup=72,72,72,72,0}
%TCIDATA{AllPages=
%F=36,\PARA{038<p type="texpara" tag="Body Text" >\hfill \thepage}
%}


\newtheorem{theorem}{Theorem}
\newtheorem{acknowledgement}[theorem]{Acknowledgement}
\newtheorem{algorithm}[theorem]{Algorithm}
\newtheorem{axiom}[theorem]{Axiom}
\newtheorem{case}[theorem]{Case}
\newtheorem{claim}[theorem]{Claim}
\newtheorem{conclusion}[theorem]{Conclusion}
\newtheorem{condition}[theorem]{Condition}
\newtheorem{conjecture}[theorem]{Conjecture}
\newtheorem{corollary}[theorem]{Corollary}
\newtheorem{criterion}[theorem]{Criterion}
\newtheorem{definition}[theorem]{Definition}
\newtheorem{example}[theorem]{Example}
\newtheorem{exercise}[theorem]{Exercise}
\newtheorem{lemma}[theorem]{Lemma}
\newtheorem{notation}[theorem]{Notation}
\newtheorem{problem}[theorem]{Problem}
\newtheorem{proposition}[theorem]{Proposition}
\newtheorem{remark}[theorem]{Remark}
\newtheorem{solution}[theorem]{Solution}
\newtheorem{summary}[theorem]{Summary}
\newenvironment{proof}[1][Proof]{\noindent\textbf{#1.} }{\ \rule{0.5em}{0.5em}}
\input{tcilatex}
\begin{document}


\section{VGFFMC}

Program to use the F-field formalizm to solve the scattering problem for a
particle of arbitrary shape. The particle is divided into an array of
dipoles on a cubic lattice (by program VGFIN). The scattering is computed
through a plane wave expansion of the field inside the particle. From this
the external field and phase function are calculated (in program VGFPHZ)

\subsubsection{Units:}

All equations are in Gausian units. Lengths are all relative, that is, if
you input a wavelength of 10 um then all other lengths must be in um. You
may use m or cm or furlongs if you wish as long as all lengths are in the
same units.

\subsubsection{Time dependence:}

The VGF code uses the time dependence $e^{-iwt)}$.

\subsubsection{Geometry:}

The particle is aligned with it's symmetry axis along the Z-axis. The
incident plane wave may be rotated around the particle by the angles alpha,
beta, and gamma (corresponding to the Euler angles as given in Arfkin,
Mathematical Methods for Physicists, 3nd Ed., Academic Press, NY, 1970). All
calculations are done in this 'body frame.' The polarization is given
relative to the x-axis in the lab or global frame. In this frame the
incident plane wave traveles in the z-hat direction and the polarization is
in the x-y plane. The direction of the electric field is measured by an
angle psi from the x-axis.

\subsubsection{History}

Author: Original code by Lines *

Last Modification by Lines 21 APR 97 *

List of subroutines and their signatures *

\subsubsection{Subroutines}

subroutine kvectors(NTH,khatN,count) *

complex Function GAM(d,k,EPS) *

complex function CPSI(R,KhatN,k,mm,N,b) *

complex Function Wcalc(X,d,k,EPS)

this

subroutine Gcalc(R,k,Gmn,d,EPS,m,n) *

real function delta(alpha,beta) *

real function dd(alpha,i,beta,k) *

subroutine ROT(RRR, alpha, beta, gamma) *

subroutine MV(M,V,U) subroutine to multiply a 3x3 matrix by a vector with
three components

\subsection{Algorithm}

We will assume a field with complex time dependance.%
\begin{equation}
\mathbf{E}\left( \mathbf{r},t\right) =\mathbf{E(r})\exp (-i\omega t)+c.c.
\label{Field}
\end{equation}%
and with a Green's function solution 
\begin{equation}
\mathbf{E}(\mathbf{r})(1+\frac{4\pi }{3}\chi )=\mathbf{E}^{e}(\mathbf{r)+}%
\int d^{3}\mathbf{r}^{\prime }\mathbf{G}\left( \mathbf{r},\mathbf{r}^{\prime
}\right) \cdot \chi \mathbf{(r}^{\prime })\mathbf{E(r}^{\prime })
\label{E-field}
\end{equation}%
We define the term%
\begin{equation}
W\left( \mathbf{r}\right) =\frac{\chi \mathbf{(r})}{(1+\frac{4\pi }{3}\chi 
\mathbf{(r}))}  \label{W1}
\end{equation}%
then letting 
\[
\mathbf{F}(\mathbf{r})=\mathbf{E}(\mathbf{r})(1+\frac{4\pi }{3}\chi ) 
\]%
and the $F$-field 
\begin{equation}
\mathbf{F}(\mathbf{r})=\mathbf{E}^{e}(\mathbf{r)+}\int d^{3}\mathbf{r}%
^{\prime }\mathbf{G}\left( \mathbf{r},\mathbf{r}^{\prime }\right) \cdot
W\left( \mathbf{r}^{\prime }\right) \mathbf{F(r}^{\prime })  \label{F-field}
\end{equation}
to make our calculations more compact.

We can write the F-field as%
\begin{equation}
F_{\mu i}=E_{\mu i}^{in}\mathbf{+}\sum\limits_{\nu }G_{\mu i\nu j}W_{\nu
}F_{\nu j}  \label{AlgF}
\end{equation}%
where, for convenience, the cell dimension $d_{\mu }^{3}$ is included in the 
$W$ term. 
\begin{equation}
W_{\nu }=Wd_{\nu }^{3}
\end{equation}%
Symbolically, this may be written as 
\begin{equation}
\mathbf{F}=\mathbf{E}^{in}+\mathbf{GWF}
\end{equation}

Proceeding directly to a numerical solution at this point would be
equivalent to the DGF. Instead, we are going to try a variational solution.
First, an error term $\mathbf{\varepsilon }$ is defined such that 
\begin{equation}
\mathbf{\varepsilon }=\mathbf{F}-\mathbf{E}^{in}-\mathbf{GWF}
\end{equation}%
Writing components yields 
\begin{equation}
\varepsilon _{\alpha i}\mathbf{=}\sum_{\beta }\sum_{j}\left( 1_{\alpha
i\beta j}\mathbf{-}G_{\alpha i\beta j}W_{\beta }\right) F_{\beta j}\mathbf{-}%
E_{\alpha i}^{in}
\end{equation}%
Here the factor 
\begin{equation}
F_{\alpha i}^{st}=-W_{\alpha }G_{\alpha i\alpha j}F_{\alpha j}  \label{AST}
\end{equation}%
is the specially defined self term discussed below as the \textquotedblleft
self term.\textquotedblright\ Note that 
\begin{equation}
1_{\alpha i\beta j}=\delta _{\alpha \beta }\delta _{ij}
\end{equation}%
Equation (\ref{ST}), gives a definition of the diagonal elements of the
dyadic Green function taken from the self term, 
\begin{equation}
F_{\alpha i}^{st}=\delta _{\alpha \beta }\left( \frac{4\pi }{3}\Gamma
_{\alpha }\right) W_{\alpha }F_{\alpha i}
\end{equation}%
Equation (\ref{Guiui}) gives a discretized form of the dyadic Green function
diagonal terms. 
\begin{equation}
G_{\alpha i\alpha j}=\frac{4\pi }{3d_{\alpha }^{3}}\Gamma _{\alpha }\delta
_{ij}
\end{equation}%
Now, for convenience, define the quantity 
\begin{equation}
M_{\alpha i\beta j}=\left( 1_{\alpha i\beta j}\mathbf{-}W_{\beta }G_{\alpha
i\beta j}\right)
\end{equation}%
where the $\alpha ,$ $i,$ $\beta ,$and $j$ are not summed over. Then 
\begin{equation}
\varepsilon _{\alpha i}\mathbf{=}M_{\alpha i\beta j}F_{\beta j}\mathbf{-}%
E_{\alpha i}^{o}  \label{M}
\end{equation}%
where in equation (\ref{M}) the summation convention has been employed
throughout, on both Latin and Greek indices.

The function $\Phi $ to extremize is defined as the squared magnitude of the
error $\mathbf{\varepsilon },$%
\begin{equation}
\Phi =\mathbf{\varepsilon }_{\alpha i}\mathbf{\varepsilon }_{\alpha i}^{\ast
}=F_{\beta j}M_{\beta j\alpha i}^{T}M_{\alpha i\gamma l}^{\ast }F_{\gamma
l}^{\ast }-E_{\alpha i}^{o\ast }M_{\alpha i\beta j}F_{\beta j}-E_{\alpha
i}^{o}M_{\alpha i\beta j}^{\ast }F_{\beta j}^{\ast }+E_{\alpha
i}^{o}E_{\alpha i}^{o\ast }  \label{PHI}
\end{equation}%
Again for convenience two additional quantities are defined which will
shorten the notation 
\begin{eqnarray}
K_{\gamma l\beta j} &=&M_{\gamma l\alpha i}^{\ast T}M_{\alpha i\beta j} \\
X_{\beta j}^{\ast } &=&E_{\alpha i}^{o\ast }M_{\alpha i\beta j}
\end{eqnarray}%
Then equation (\ref{PHI}) can be written compactly as 
\begin{equation}
\Phi =F_{\gamma l}^{\ast }K_{\gamma l\beta j}F_{\beta j}-\left[ X_{\beta
j}^{\ast }F_{\beta j}+X_{\beta j}F_{\beta j}^{\ast }\right] +E_{\alpha
i}^{o}E_{\alpha i}^{o\ast }
\end{equation}%
Because the form of $\mathbf{F}$ is not known, a trial function $\mathbf{%
\tilde{F}}$ can be used which is an expansion in some basis function set $%
\Psi _{\beta N}.$ We define the trial function $\mathbf{\tilde{F}}$ by 
\begin{equation}
\tilde{F}_{\beta j}\mathbf{=}\sum_{N=1}^{N_{k}}a_{Nj}\Psi _{\beta N}
\end{equation}%
where the sum has been written explicitly. The function $\mathbf{\tilde{F}}$
is substituted for $\mathbf{F.}$ The functions $\Psi _{\beta N}$ we will
assume to be of the form

\begin{equation}
\Psi _{\beta N}=e^{imk\mathbf{\hat{k}}_{N}\mathbf{\cdot r}_{\beta }}
\end{equation}%
which depend on the $\mathbf{\hat{k}}_{N}$

These are read in from a file. That file is created by vgfkv.f. And it looks
like it makes a $k_{N}$ in equally spaced increments in the $\theta $ and $%
\phi $ directions. \FRAME{ftbpF}{2.904in}{2.7043in}{0pt}{}{}{Figure}{\special%
{language "Scientific Word";type "GRAPHIC";maintain-aspect-ratio
TRUE;display "USEDEF";valid_file "T";width 2.904in;height 2.7043in;depth
0pt;original-width 3.9418in;original-height 3.6668in;cropleft "0";croptop
"1";cropright "1";cropbottom "0";tempfilename
'QSLJDD00.wmf';tempfile-properties "XPR";}}Then equation (\ref{PHI}) is
approximately given by 
\begin{equation}
\Phi =\Psi _{\gamma M}^{\ast }a_{Ml}^{\ast }K_{\gamma l\beta j}a_{Nj}\Psi
_{\beta N}-\left[ X_{\beta j}^{\ast }\Psi _{\beta N}a_{Nj}+X_{\beta j}\Psi
_{\beta N}^{\ast }a_{Nj}^{\ast }\right] +E_{\alpha i}^{o}E_{\alpha i}^{o\ast
}
\end{equation}%
Now the notation can again be simplified by defining two functions 
\begin{eqnarray}
H_{MlNj} &=&\Psi _{\gamma M}^{\ast }K_{\gamma l\beta j}\Psi _{\beta N} \\
Y_{Nj}^{\ast } &=&\Psi _{N\beta }^{T}X_{\beta j}^{\ast }
\end{eqnarray}%
Then the final form of equation (\ref{PHI}) is 
\begin{equation}
\Phi =a_{Ml}^{\ast }H_{MlNj}a_{Nj}-\left[ Y_{Nj}^{\ast
}a_{Nj}+Y_{Nj}a_{Nj}^{\ast }\right] +E_{\alpha i}^{o}E_{\alpha i}^{o\ast }
\label{PhiSimp}
\end{equation}%
To extremize $\Phi ,$ equation (\ref{PhiSimp}) is differentiated with
respect to $a_{Li}^{\ast },$ yielding 
\begin{equation}
\frac{\partial \Phi }{\partial a_{Li}^{\ast }}=H_{LiNj}a_{Nj}-Y_{Li}
\label{DerPhi}
\end{equation}%
The result is set equal to zero. Equation (\ref{DerPhi}) can be written in
matrix form as 
\begin{equation}
\left( \mathbf{H}\right) \left( \mathbf{a}\right) =\left( \mathbf{Y}\right)
\label{HaY}
\end{equation}%
The solution for $\left( \mathbf{a}\right) $ is given by 
\begin{equation}
\left( \mathbf{a}\right) =\left( \mathbf{H}\right) ^{-1}\left( \mathbf{Y}%
\right)  \label{aHY1}
\end{equation}

Writing out the $\left( \mathbf{H}\right) $and $(\mathbf{Y})$ matrix
explicitly in components we have 
\begin{eqnarray}
H_{MlNj} &=&\Psi _{\gamma M}^{\ast }\left( 1_{\alpha i\gamma l}\mathbf{-}%
W_{\beta }^{\ast }G_{\alpha i\gamma l}^{\ast }\right) \left( 1_{\alpha
i\beta j}\mathbf{-}W_{\beta }G_{\alpha i\beta j}\right) \Psi _{\beta N} 
\nonumber \\
Y_{Nj} &=&\left( 1_{\alpha i\beta j}\mathbf{-}W_{\beta }G_{\alpha i\beta
j}\right) ^{\ast }E_{\alpha i}^{o}\Psi _{\beta N}^{\ast }
\end{eqnarray}%
The functions $\Psi _{\beta N}$ could be any set of functions. A simple
choice for larger particles with refractive indices that are not very
different from unity is a set of plane waves: 
\begin{equation}
\Psi _{\beta N}=e^{imk\mathbf{\hat{k}}_{N}\mathbf{\cdot r}_{\beta }}
\end{equation}%
where as before $m$ is the complex refractive index of the particle. These
functions obey the correct wave equation inside the particle. The trial
functions $\tilde{F}_{\beta j}$ are given by 
\begin{equation}
\tilde{F}_{\beta j}\mathbf{=}\sum_{N=1}^{N_{k}}a_{Nj}e^{imk\mathbf{\hat{k}}%
_{N}\mathbf{\cdot r}_{\beta }}  \label{Ftilde}
\end{equation}%
Again, the summation has been written explicitly. The $\mathbf{\hat{k}}_{N}$
or \textquotedblleft $k$-vectors\textquotedblright\ must be chosen to
represent the internal wave well. A \textquotedblleft goodness of
choice\textquotedblright\ criterion for the number and direction of the $%
\mathbf{\hat{k}}_{N}$ vectors results from evaluating $\Phi $ explicitly.
Observing the values of the $a_{Nj}$ can determine which of the $k$-vectors
are not contributing well. If the magnitude of an $a_{N}$ is small, then the 
$\mathbf{\hat{k}}_{N}$ is not well placed or may not be needed.

For the sake of computation, a temporary variable is employed in the
calculations that follow. 
\begin{equation}
T_{\alpha iNj}=\sum_{\beta }\left( 1_{\alpha i\beta j}\mathbf{-}G_{\alpha
i\beta j}W_{\beta }\right) \Psi _{\beta N}
\end{equation}
Using $T_{\alpha iNj},$ $H_{MlNj}$ and $Y_{Nj}$ can be expressed as

\begin{eqnarray}
H_{MlNj} &=&\sum_{\alpha }\sum_{i}T_{\alpha iMl}^{\ast }T_{\alpha iNj} \\
Y_{Nj} &=&\sum_{\alpha }\sum_{i}T_{\alpha iNj}^{\ast }E_{\alpha i}^{o} 
\nonumber
\end{eqnarray}%
This is the form of the quantities used in the actual algorithms.

Then since we foudn the $a_{Nj}$ with the matrix inversion we can calculate
the 
\begin{equation}
\tilde{F}_{\beta j}\mathbf{=}\sum_{N=1}^{N_{k}}a_{Nj}e^{imk\mathbf{\hat{k}}%
_{N}\mathbf{\cdot r}_{\beta }}
\end{equation}%
and these are related to the electric field like this

\[
\mathbf{F}(\mathbf{r})=\mathbf{E}(\mathbf{r})(1+\frac{4\pi }{3}\chi ) 
\]%
so 
\[
\frac{\mathbf{F}(\mathbf{r})}{(1+\frac{4\pi }{3}\chi )}=\mathbf{E}(\mathbf{r}%
) 
\]%
and we have our electric field.

This is what the code is supposed to do.

\subsection{Self Term}

Equation (\ref{AlgF}) can be written as a sum where the infinitesimal volume 
$d^{3}r^{\prime }$ is now a finite volume $d_{\nu }^{3}$:

\begin{eqnarray}
F_{i}(\mathbf{r}_{\mu }) &=&E_{i}^{in}\left( \mathbf{r}_{\mu }\right) 
\mathbf{+}\sum\limits_{\nu \neq \mu }d_{\nu }^{3}G_{ij}\left( \mathbf{r}%
_{\mu \nu }\right) WF_{j}\mathbf{(r}_{\nu }^{\prime })+I_{i}(\mathbf{r}_{\mu
}) \\
&=&E_{\mu ,i}^{in}\mathbf{+}W\sum\limits_{\nu \neq \mu }d_{\nu }^{3}G_{\mu
i\nu j}F_{\nu j}+I_{\mu i}  \nonumber
\end{eqnarray}%
Here $I_{\mu i}$ is the self term, and Greek indices run and sum over dipole
cell number and the Latin indices run and sum over vector components. Also, $%
\mathbf{r}_{\mu }$ is at the center of cell number $\mu .$ Of course, this
only works well if $d_{\mu }^{3}$ is small. Experience shows that $%
\left\vert m\right\vert kd<1$ should be used, where $m$ is the complex
refractive index of the homogenous particle.

The Green's function is discretized as 
\begin{equation}
G_{\mu i\nu j}=\exp (ikR_{\mu \nu })\left[ \frac{k^{2}}{R_{\mu \nu }}\left(
\delta _{ij}-\widehat{R_{\mu \nu i}}\widehat{R_{\mu \nu j}}\right) +\left( 
\frac{ik}{R_{\mu \nu }^{2}}-\frac{1}{R_{\mu \nu }^{3}}\right) \left( \delta
_{ij}-3\widehat{R_{\mu \nu i}}\widehat{R_{\mu \nu j}}\right) \right]
\end{equation}
where $\mathbf{R}_{\mu \nu }=\mathbf{r}_{\mu }-\mathbf{r}_{\nu }.$ This
equation is very similar to that used by Goedecke and O'Brien.\cite%
{Goedecke88}

But a problem occurs when the summation index $\nu $ is equal to $\mu :$
Terms with $R_{\mu \mu }$ in the denominator would diverge (blow up to
infinity). Thus in equation (\ref{F1}) these terms have been separated.
Because such a term results from the action of the cell field on itself, it
is known as the self-term. The self-term results from letting $\mathbf{r}=%
\mathbf{r}_{\mu },$ and $\mathbf{r}^{\prime }=\mathbf{r}_{\mu }+\mathbf{\xi }
$, in equation (\ref{F-field}) , and taking $\chi \left( \mathbf{r}^{\prime
}\right) \mathbf{F}_{j}\left( \mathbf{r}^{\prime }\right) =\chi \left( 
\mathbf{r}_{\mu }\right) \mathbf{F}_{j}\left( \mathbf{r}_{\mu }\right) .$
One gets 
\begin{equation}
I_{\mu ,i}=\left[ \int_{\Box }d^{3}\xi G_{ij}\left( \mathbf{\xi }\right) %
\right] WF_{j}\mathbf{(r}_{\mu })
\end{equation}

The $\Box $ indicates that the integral is over the small dipole cell, of
volume $d_{\mu }^{3}.$ The integral in brackets in equation (\ref{SelfTerm})
can be written as 
\begin{equation}
\int_{\Box }d^{3}\xi G_{ij}\left( \mathbf{\xi }\right) =\int_{\Box }d^{3}\xi
e^{ik\xi }\left[ \frac{k^{2}}{\xi }\left( \delta _{ij}-\hat{\xi}_{i}\hat{\xi}%
_{j}\right) +\left( \frac{ik}{\xi ^{2}}-\frac{1}{\xi ^{3}}\right) \left(
\delta _{ij}-3\hat{\xi}_{i}\hat{\xi}_{j}\right) \right]
\end{equation}
Employing the identity 
\begin{equation}
\int_{\Box }d\Omega _{\xi }\hat{\xi}_{i}\hat{\xi}_{j}=\delta _{ij}\frac{1}{3}%
\int_{\Box }d\Omega _{\xi }
\end{equation}
the last term vanishes due to symmetry of cubical cells. Also, the remaining
terms can be combined to yield 
\begin{equation}
\int_{\Box }d^{3}\xi G_{ij}\left( \mathbf{\xi }\right) =\delta _{ij}\frac{%
2k^{2}}{3}\int_{\Box }d^{3}\xi \frac{k^{2}e^{ik\xi }}{\xi }
\end{equation}

A series expansion of the integrand on the right hand side yields

\begin{equation}
\frac{\exp (ikx)}{x}=\left[ \frac{1}{x}+ik-\frac{1}{2}k^{2}x-\frac{1}{6}%
ik^{3}x^{2}+\frac{1}{24}k^{4}x^{3}+Ox^{4}\right]
\end{equation}%
which can be substituted into the previous expression to yield 
\begin{eqnarray}
\int_{\Box }d^{3}\xi G_{ij}\left( \mathbf{\xi }\right) &=&\delta _{ij}\frac{%
2k^{2}}{3}\int_{\Box }d^{3}\xi \left( \frac{1}{\xi }+ik-\frac{1}{2}k^{2}\xi -%
\frac{1}{6}ik^{3}\xi ^{2}+\frac{1}{24}k^{4}\xi ^{3}+O\left( \xi ^{4}\right)
\right)  \nonumber \\
&\equiv &\delta _{ij}\frac{4\pi }{3}\Gamma
\end{eqnarray}%
where this equation defines $\Gamma $ as 
\begin{equation}
\Gamma =\left( \left( ka\right) ^{2}+\allowbreak \frac{2}{3}\allowbreak
i\left( ka\right) ^{3}-\frac{1}{4}\left( ka\right) ^{4}-\frac{1}{15}i\left(
ka\right) ^{5}+O\left( ka\right) ^{6}\right)
\end{equation}%
and the integral has been done by using an equivalent volume sphere with
radius $a_{\mu }$ 
\begin{equation}
a_{\mu }=d_{\mu }\left( \frac{3}{4\pi }\right) ^{\frac{1}{3}}
\end{equation}%
thus 
\begin{eqnarray}
\Gamma _{\mu } &=&\left( \frac{3}{4\pi }\right) ^{\frac{2}{3}}\left( kd_{\mu
}\right) ^{2}+\allowbreak \allowbreak \left( \frac{\allowbreak \allowbreak i%
}{2\pi }\right) \left( kd_{\mu }\right) ^{3}-\frac{1}{4}\left( \frac{3}{4\pi 
}\right) ^{\frac{4}{3}}\left( kd_{\mu }\right) ^{4}  \label{Gamma1} \\
&&-\frac{1}{15}i\left( \frac{3}{4\pi }\right) ^{\frac{5}{3}}\left( kd_{\mu
}\right) ^{5}+O\left( \left( kd_{\mu }\right) ^{6}\right)  \nonumber
\end{eqnarray}%
The expression for $I_{\mu i}$ can then be written as

\begin{equation}
I_{\mu i}=\delta _{ij}\frac{4\pi }{3}W\Gamma _{\mu }F_{\mu j}=\frac{4\pi }{3}%
W\Gamma _{\mu }F_{\mu i}  \label{ST1}
\end{equation}

The first two terms in $\Gamma _{\mu }$ were used by Goedecke and O'Brien in
their formulation of the DGF scattering code \cite{Goedecke88}. Draine and
Goodman obtained a somewhat different expression for $\Gamma _{\mu }$ for
placement of the dipole cells on a cubic lattice\cite{Draine93}, i.e. the
coefficients of the various powers of $kd_{\mu }$ are different than in
equation(\ref{Gamma}). However, for ease in comparison with the DGF method,
the form given by Goedecke and O'Brien is used here. The corrections of
order $\left( kd_{\mu }\right) ^{4}$and higher are insignificant; and even
the term in $\left( kd_{\mu }\right) ^{2}$ is not important if $kd_{\mu }\ll
1.$ But, as shown by Goedecke and O'Brien, the lowest order imaginary term
proportional to $\left( kd_{\mu }\right) ^{3}$ is essential for agreement
with the optical theorem, and this term is the same in the Draine and
Goodman expression.

\subsection{Initialized variables}

The code initializes variables in lines 51-86. ForTran requires variables to
be declared like in C++. It uses parenthesis to indicate an array.

We start with 
\[
\begin{tabular}{|l|l|}
\hline
the maximum number of possible dipole cells & $NMAX$ \\ \hline
Number of $k$-vectors & $KNMAX$ \\ \hline
\end{tabular}%
\]

\[
\begin{tabular}{|l|l|}
\hline
& $IPVT(3\ast KNMAX)$ \\ \hline
& $z\left( 3\ast KNMAX\right) $ \\ \hline
& $rcond$ \\ \hline
& $NUSE$ \\ \hline
& $I$ \\ \hline
& $J$ \\ \hline
& $mcount$ \\ \hline
& $kcount$ \\ \hline
\end{tabular}%
\]%
\[
\begin{tabular}{|l|l|}
\hline
& $N$ \\ \hline
& $M$ \\ \hline
& $l$ \\ \hline
& $a$ \\ \hline
& $b$ \\ \hline
& $NK$ \\ \hline
& $np$ \\ \hline
& $mp$ \\ \hline
& $iseed$ \\ \hline
& $ikcount$ \\ \hline
& $itemp$ \\ \hline
\end{tabular}%
\]%
\[
\begin{tabular}{|l|l|}
\hline
& $N$ \\ \hline
& $M$ \\ \hline
& $l$ \\ \hline
& $a$ \\ \hline
& $b$ \\ \hline
& $NK$ \\ \hline
& $np$ \\ \hline
& $mp$ \\ \hline
& $iseed$ \\ \hline
& $ikcount$ \\ \hline
& $itemp$ \\ \hline
\end{tabular}%
\]

\[
\begin{tabular}{|l|l|}
\hline
wavelength & $Wave$ \\ \hline
number of dipoles in along the major axis & $NSID$ \\ \hline
fine structure divisions per dipole cell side & $NLSID$ \\ \hline
Particle symmetry semi-axis & $RAD$ \\ \hline
Incident direction & $\left( \alpha lpha,\beta eta,gamma\right) $ \\ \hline
polarization angle & $psi$ \\ \hline
real and imaginary parts of index of refraction & $MR,$ $MI$ \\ \hline
index of refraction (complex) & $M$ \\ \hline
$\pi $ & $PI$ \\ \hline
Aspect ratio, ration of major to minor axis & $AR$ \\ \hline
\end{tabular}%
\]%
\[
\begin{tabular}{|l|l|}
\hline
Dipole weighting factors & $D(NMAX)$ \\ \hline
Locations of dipoles, $x,$ $y,$ and $z$ components. & $R(NMAX,3)$ \\ \hline
real and imaginary part of the permittivity & $ER,$ $EI$ \\ \hline
permitivity of the particle material & $EPS$ \\ \hline
$r\cdot k$ & $RDK$ \\ \hline
& $TD$ \\ \hline
wave number $2\pi /\lambda $ & $k$ \\ \hline
& $PI$ \\ \hline
Convert from degrees to radioas & $DEG$ \\ \hline
& $Khat\left( 3\right) $ \\ \hline
Rotation matrix & $RRR\left( 3,3\right) $ \\ \hline
& $V\left( 3\right) $ \\ \hline
I think this is the input electric field direction & $E0hat\left( 3\right) $
\\ \hline
\end{tabular}%
\]%
\[
\begin{tabular}{|l|l|}
\hline
& $KhatN\left( KNMAX,3\right) $ \\ \hline
& $dtemp$ \\ \hline
& $dsum$ \\ \hline
& $divd$ \\ \hline
& $kth\left( KNMAX\right) $ \\ \hline
& $kph\left( KNMAX\right) $ \\ \hline
& $ERR$ \\ \hline
& $ERRlast$ \\ \hline
& $ERR0$ \\ \hline
\end{tabular}%
\]%
\[
\begin{tabular}{|l|l|}
\hline
& $aa\left( 3\ast KNMAX,3\ast KNMAX\right) $ \\ \hline
& $bb\left( 3\ast KNMAX\right) $ \\ \hline
complex index of refraction (21) & $mm$ \\ \hline
complex permittivity (20) & $EPS$ \\ \hline
complex susceptibility (20) & $X$ \\ \hline
W-factor (39) $\chi \left( r\right) /(1+\frac{4}{3}\pi \chi \left( r\right)
) $ & $W$ \\ \hline
& $C$ \\ \hline
& $CI$ \\ \hline
& $temp$ \\ \hline
\end{tabular}%
\]%
\[
\begin{tabular}{|l|l|}
\hline
Incident field (line 138-148) & $E0\left( 3\ast KNMAX\right) $ \\ \hline
& $E\left( 3\ast KNMAX\right) $ \\ \hline
& $T1\left( NMAX,3\ast KNMAX,3\right) $ \\ \hline
& $F\left( NMAX,3\right) $ \\ \hline
& $H(KNMAX,3,KNMAX,3)$ \\ \hline
& $Y(KNMAX,3)$ \\ \hline
& $An(KNMAX,3)$ \\ \hline
& $PHI$ \\ \hline
\end{tabular}%
\]

\[
\begin{tabular}{ll}
& $PI=3.141592654$ \\ 
& $DEG=PI/180.0$ \\ 
& $CI=(0.0,1.0)$ \\ 
& $iseed=234564$ \\ 
&  \\ 
&  \\ 
& 
\end{tabular}%
\]

\section{Main Program}

The program defines many variables at the start. After defining $\pi $ and $%
i $ in line $86,$ it defines the particle permittivity $\epsilon $ in line
107, the complex index of refraction $mm$ in line $108,$ and the complex
suseptability $\chi $ in line 109.

It inputs the positions and weights for our dipoles in lines 110 to 116.

It then find the wave number, $k$ in line 119 and it rotates the direction
of the incident electric field into the position we asked for in vgfin in
lines 125-135.

It then calculates the $W$ factor (137) 
\[
W\left( \nu \right) =\frac{\chi }{1+\frac{4\pi }{3}\chi } 
\]%
and the incident feild (147) 
\[
E_{o}\left( r\right) =e^{-ik\mathbf{r}\cdot \mathbf{\hat{k}}} 
\]%
for every $r$ where we have placed a dipole.

The program then calls a subroutine to bring in our direction angles for
each of the $k-vectors.$ (152)

The program then calles kvector3 to turn the $k$-vector direction angles
into the components of the $\hat{k}$ vectors.

The program then starts the monte carlo loop. In this loop it calclates
where 
\begin{equation}
\Psi _{\beta N}=e^{imk\mathbf{\hat{k}}_{N}\mathbf{\cdot r}_{\beta }}
\label{PSYFUN}
\end{equation}%
\[
\psi =e^{(i\ast k\ast khatN.R(b))} 
\]%
equation 183

We want to find the $a_{Nj}$ in 
\begin{equation}
\tilde{F}_{\beta j}\mathbf{=}\sum_{N=1}^{N_{k}}a_{Nj}\Psi _{\beta N}
\end{equation}%
and use this to find our $E$ field%
\[
\frac{\mathbf{F}(\mathbf{r})}{(1+\frac{4\pi }{3}\chi )}=\mathbf{E}(\mathbf{r}%
) 
\]

The code caluclates 
\[
T_{\alpha iNj}=\sum_{\beta }\left( 1_{\alpha i\beta j}\mathbf{-}G_{\alpha
i\beta j}W_{\beta }\right) \Psi _{\beta N} 
\]%
\[
T1_{(a,i,N,j)}=T1_{(a,i,N,j)}+(dd_{(a,i,b,j)}-d_{(b)}^{3}\ast W\ast GG)\psi 
\]

\[
T1_{(a,i,N,j)}=T1_{(a,i,N,j)}+(dd_{(a,i,b,j)}-d_{(b)}^{3}\ast W\ast
GG_{(R,k,dtemp,EPS,a,i,b,j)})e^{(i\ast k\ast khatN.R(b))} 
\]

which is equation 185 and happens between lines (166) and (181).

@@@I don't know what $d\left( b\right) $ is yet.

Now it calculates Y (191)%
\[
Y_{Nj}=\sum_{\alpha }\sum_{i}T_{\alpha iNj}^{\ast }E_{\alpha i}^{o} 
\]%
and then (199)

\[
H_{MlNj}=\sum_{\alpha }\sum_{i}T_{\alpha iMl}^{\ast }T_{\alpha iNj} 
\]

Then it does a matrix solve. It first must reshape our multidimensional
matricies into two-dimensional matricies because that is what the inversion
routine can do.

So $Y_{Nj}$ is turned into $bb\left( m\right) $ and $H_{MlNj}$ is turned
into $aa(m,mp)$

It then does something with ipvt and z that I\ need to figure out @@@@@

The call to cgeco Factors a COMPLEX matrix by Gaussian elimination and
estimates the condition of the matrix.

The call to Solves the COMPLEX system A*X=B or CTRANS(A)*X=B using the
factors computed by CGECO or CGEFA. The verablee $bb$ is the solution.

The result should be the $a_{Nj}$ values which we then use to calculate 
\begin{equation}
\tilde{F}_{\beta j}\mathbf{=}\sum_{N=1}^{N_{k}}a_{Nj}\Psi _{\beta N}
\end{equation}%
and then 
\[
\mathbf{E}(\mathbf{r})=\frac{\mathbf{F}(\mathbf{r})}{(1+\frac{4\pi }{3}\chi )%
} 
\]

\section{Subroutines}

\subsection{Subroutine
getkvectors(kth,kph,NK,KFILE,ERR,ERRlast,mcount,kcount)}

subroutine to read in the kvectkors from the file KFILE

subroutine kvectors(NTH,khatN,count) *

complex Function GAM(d,k,EPS) *

complex function CPSI(R,KhatN,k,mm,N,b) *

\subsection{complex Function Wcalc(X,d,k,EPS)}

this fucntion calcualtes the W-factor from the dissertation equation 39 and
141.%
\[
W\left( \nu \right) =\frac{\chi }{1+\frac{4\pi }{3}\chi } 
\]%
where $\chi $ is the complex suseptibility.

subroutine Gcalc(R,k,Gmn,d,EPS,m,n) *

real function delta(alpha,beta) *

real function dd(alpha,i,beta,k) *

subroutine ROT(RRR, alpha, beta, gamma) *

subroutine MV(M,V,U) subroutine to multiply a 3x3 matrix by a vector with
three components

\subsection{The digitized Green's function}

Equation (\ref{F-field}) can be written as a sum where the infinitesimal
volume $d^{3}r^{\prime }$ is now a finite volume $d_{\nu }^{3}$:

\begin{eqnarray}
F_{i}(\mathbf{r}_{\mu }) &=&E_{i}^{in}\left( \mathbf{r}_{\mu }\right) 
\mathbf{+}\sum\limits_{\nu \neq \mu }d_{\nu }^{3}G_{ij}\left( \mathbf{r}%
_{\mu \nu }\right) WF_{j}\mathbf{(r}_{\nu }^{\prime })+I_{i}(\mathbf{r}_{\mu
})  \label{F1} \\
&=&E_{\mu ,i}^{in}\mathbf{+}W\sum\limits_{\nu \neq \mu }d_{\nu }^{3}G_{\mu
i\nu j}F_{\nu j}+I_{\mu i}  \nonumber
\end{eqnarray}
Here $I_{\mu i}$ is the self term, and Greek indices run and sum over dipole
cell number and the Latin indices run and sum over vector components. Also, $%
\mathbf{r}_{\mu }$ is at the center of cell number $\mu .$ Equation (\ref{F1}%
) only approximates equation (\ref{F-field}) well if $d_{\mu }^{3}$ is
small. Experience shows that $\left| m\right| kd<1$ should be used, where $m$
is the complex refractive index of the homogenous particle.

The Green's function is discretized as 
\begin{equation}
G_{\mu i\nu j}=\exp (ikR_{\mu \nu })\left[ \frac{k^{2}}{R_{\mu \nu }}\left(
\delta _{ij}-\widehat{R_{\mu \nu i}}\widehat{R_{\mu \nu j}}\right) +\left( 
\frac{ik}{R_{\mu \nu }^{2}}-\frac{1}{R_{\mu \nu }^{3}}\right) \left( \delta
_{ij}-3\widehat{R_{\mu \nu i}}\widehat{R_{\mu \nu j}}\right) \right]
\end{equation}
where $\mathbf{R}_{\mu \nu }=\mathbf{r}_{\mu }-\mathbf{r}_{\nu }.$ This
equation is very similar to that used by Goedecke and O'Brien.\cite%
{Goedecke88}

\subsubsection{Self-Term}

A problem occurs when the summation index $\nu $ is equal to $\mu :$ Terms
with $R_{\mu \mu }$ in the denominator would diverge. Thus in equation (\ref%
{F1}) these terms have been separated. Because such a term results from the
action of the cell field on itself, it is known as the self-term. The
self-term results from letting $\mathbf{r}=\mathbf{r}_{\mu },$ and $\mathbf{r%
}^{\prime }=\mathbf{r}_{\mu }+\mathbf{\xi }$, in equation (\ref{F-field}) ,
and taking $\chi \left( \mathbf{r}^{\prime }\right) \mathbf{F}_{j}\left( 
\mathbf{r}^{\prime }\right) =\chi \left( \mathbf{r}_{\mu }\right) \mathbf{F}%
_{j}\left( \mathbf{r}_{\mu }\right) .$ One gets 
\begin{equation}
I_{\mu ,i}=\left[ \int_{\Box }d^{3}\xi G_{ij}\left( \mathbf{\xi }\right) %
\right] WF_{j}\mathbf{(r}_{\mu })  \label{SelfTerm}
\end{equation}

The $\Box $ indicates that the integral is over the small dipole cell, of
volume $d_{\mu }^{3}.$ The integral in brackets in equation (\ref{SelfTerm})
can be written as 
\begin{equation}
\int_{\Box }d^{3}\xi G_{ij}\left( \mathbf{\xi }\right) =\int_{\Box }d^{3}\xi
e^{ik\xi }\left[ \frac{k^{2}}{\xi }\left( \delta _{ij}-\hat{\xi}_{i}\hat{\xi}%
_{j}\right) +\left( \frac{ik}{\xi ^{2}}-\frac{1}{\xi ^{3}}\right) \left(
\delta _{ij}-3\hat{\xi}_{i}\hat{\xi}_{j}\right) \right]
\end{equation}
Employing the identity 
\begin{equation}
\int_{\Box }d\Omega _{\xi }\hat{\xi}_{i}\hat{\xi}_{j}=\delta _{ij}\frac{1}{3}%
\int_{\Box }d\Omega _{\xi }
\end{equation}
the last term vanishes due to symmetry of cubical cells. Also, the remaining
terms can be combined to yield 
\begin{equation}
\int_{\Box }d^{3}\xi G_{ij}\left( \mathbf{\xi }\right) =\delta _{ij}\frac{%
2k^{2}}{3}\int_{\Box }d^{3}\xi \frac{k^{2}e^{ik\xi }}{\xi }
\end{equation}

A series expansion of the integrand on the right hand side yields

\begin{equation}
\frac{\exp (ikx)}{x}=\left[ \frac{1}{x}+ik-\frac{1}{2}k^{2}x-\frac{1}{6}%
ik^{3}x^{2}+\frac{1}{24}k^{4}x^{3}+Ox^{4}\right]
\end{equation}
which can be substituted into the previous expression to yield 
\begin{eqnarray}
\int_{\Box }d^{3}\xi G_{ij}\left( \mathbf{\xi }\right) &=&\delta _{ij}\frac{%
2k^{2}}{3}\int_{\Box }d^{3}\xi \left( \frac{1}{\xi }+ik-\frac{1}{2}k^{2}\xi -%
\frac{1}{6}ik^{3}\xi ^{2}+\frac{1}{24}k^{4}\xi ^{3}+O\left( \xi ^{4}\right)
\right)  \nonumber \\
&\equiv &\delta _{ij}\frac{4\pi }{3}\Gamma
\end{eqnarray}
where this equation defines $\Gamma $ as 
\begin{equation}
\Gamma =\left( \left( ka\right) ^{2}+\allowbreak \frac{2}{3}\allowbreak
i\left( ka\right) ^{3}-\frac{1}{4}\left( ka\right) ^{4}-\frac{1}{15}i\left(
ka\right) ^{5}+O\left( ka\right) ^{6}\right)
\end{equation}
and the integral has been done by using an equivalent volume sphere with
radius $a_{\mu }$ 
\begin{equation}
a_{\mu }=d_{\mu }\left( \frac{3}{4\pi }\right) ^{\frac{1}{3}}
\end{equation}
thus 
\begin{eqnarray}
\Gamma _{\mu } &=&\left( \frac{3}{4\pi }\right) ^{\frac{2}{3}}\left( kd_{\mu
}\right) ^{2}+\allowbreak \allowbreak \left( \frac{\allowbreak \allowbreak i%
}{2\pi }\right) \left( kd_{\mu }\right) ^{3}-\frac{1}{4}\left( \frac{3}{4\pi 
}\right) ^{\frac{4}{3}}\left( kd_{\mu }\right) ^{4}  \label{Gamma} \\
&&-\frac{1}{15}i\left( \frac{3}{4\pi }\right) ^{\frac{5}{3}}\left( kd_{\mu
}\right) ^{5}+O\left( \left( kd_{\mu }\right) ^{6}\right)  \nonumber
\end{eqnarray}
The expression for $I_{\mu i}$ can then be written as

\begin{equation}
I_{\mu i}=\delta _{ij}\frac{4\pi }{3}W\Gamma _{\mu }F_{\mu j}=\frac{4\pi }{3}%
W\Gamma _{\mu }F_{\mu i}  \label{ST}
\end{equation}

The first two terms in $\Gamma _{\mu }$ were used by Goedecke and O'Brien in
their formulation of the DGF scattering code \cite{Goedecke88}. Draine and
Goodman obtained a somewhat different expression for $\Gamma _{\mu }$ for
placement of the dipole cells on a cubic lattice\cite{Draine93}, i.e. the
coefficients of the various powers of $kd_{\mu }$ are different than in
equation(\ref{Gamma}). However, for ease in comparison with the DGF method,
the form given by Goedecke and O'Brien is used here. The corrections of
order $\left( kd_{\mu }\right) ^{4}$and higher are insignificant; and even
the term in $\left( kd_{\mu }\right) ^{2}$ is not important if $kd_{\mu }\ll
1.$ But, as shown by Goedecke and O'Brien, the lowest order imaginary term
proportional to $\left( kd_{\mu }\right) ^{3}$ is essential for agreement
with the optical theorem, and this term is the same in the Draine and
Goodman expression.

The code that does this is the subroutine GG

\bigskip

%TCIMACRO{%
%\TeXButton{GG code}{\begin{verbatim}
%C***********************************************************************
%C*---------------------------------------------------------------------*
%complex function GG(R,k,d,EPS,a,i,b,j)      
%C*---------------------------------------------------------------------*      
%C***********************************************************************
%C*    Function to calculate the dyadic Green's function for a dipole   *
%C*      in the IBM write-up (equation ???)                             *
%C*    single value checked 1 Aug 96  Formula Checked 6 Aug 96          *
%C***********************************************************************
%C *** Set the value of NMAX via an included file                     ***  
%implicit none
%include 'nmax.inc'
%C****  Variables                                                    ****
%complex PHZ,t1,t2,temp,CI,GAM,EPS
%real RMAG,Rhat(3),R(NMAX,3),Rab(3),k,K2,d,PI
%real delta
%integer a,b,i,j
%Parameter(PI=3.141592654,CI=(0.0,1.0))
%C
%K2=k*k 
%C       d3=d**3
%if(b.ne.a) then
%C         calculate separation distance Rmn=Rn-Rm and RMAG=|Rmn|
%  Rab(1)=R(a,1)-R(b,1)
%  Rab(2)=R(a,2)-R(b,2)
%  Rab(3)=R(a,3)-R(b,3)
%  RMAG=Rab(1)**2+Rab(2)**2+Rab(3)**2    
%  RMAG=RMAG**0.5
%C         Make a unit vector in the Rmn direction                    ***
%  Rhat(1)=Rab(1)/RMAG 
%  Rhat(2)=Rab(2)/RMAG
%  Rhat(3)=Rab(3)/RMAG
%C                          
%  temp=CI*k*RMAG  
%  PHZ=cexp(temp) 
%C         
%  t1=(K2/RMAG)*(delta(i,j)-Rhat(i)*Rhat(j)) 
%  t2=(ci*k/RMAG**2-1.0/RMAG**3)*(delta(i,j)-3.*Rhat(i)*Rhat(j))
%  GG=PHZ*(t1+t2)  
%else
%  GG=4.*PI*GAM(d,k,EPS)/(3.0*d**3)
%end if        
%return
%end
%\end{verbatim}}}%
%BeginExpansion
\begin{verbatim}
C***********************************************************************
C*---------------------------------------------------------------------*
complex function GG(R,k,d,EPS,a,i,b,j)      
C*---------------------------------------------------------------------*      
C***********************************************************************
C*    Function to calculate the dyadic Green's function for a dipole   *
C*      in the IBM write-up (equation ???)                             *
C*    single value checked 1 Aug 96  Formula Checked 6 Aug 96          *
C***********************************************************************
C *** Set the value of NMAX via an included file                     ***  
implicit none
include 'nmax.inc'
C****  Variables                                                    ****
complex PHZ,t1,t2,temp,CI,GAM,EPS
real RMAG,Rhat(3),R(NMAX,3),Rab(3),k,K2,d,PI
real delta
integer a,b,i,j
Parameter(PI=3.141592654,CI=(0.0,1.0))
C
K2=k*k 
C       d3=d**3
if(b.ne.a) then
C         calculate separation distance Rmn=Rn-Rm and RMAG=|Rmn|
  Rab(1)=R(a,1)-R(b,1)
  Rab(2)=R(a,2)-R(b,2)
  Rab(3)=R(a,3)-R(b,3)
  RMAG=Rab(1)**2+Rab(2)**2+Rab(3)**2    
  RMAG=RMAG**0.5
C         Make a unit vector in the Rmn direction                    ***
  Rhat(1)=Rab(1)/RMAG 
  Rhat(2)=Rab(2)/RMAG
  Rhat(3)=Rab(3)/RMAG
C                          
  temp=CI*k*RMAG  
  PHZ=cexp(temp) 
C         
  t1=(K2/RMAG)*(delta(i,j)-Rhat(i)*Rhat(j)) 
  t2=(ci*k/RMAG**2-1.0/RMAG**3)*(delta(i,j)-3.*Rhat(i)*Rhat(j))
  GG=PHZ*(t1+t2)  
else
  GG=4.*PI*GAM(d,k,EPS)/(3.0*d**3)
end if        
return
end
\end{verbatim}%
%EndExpansion

which calles the subroutine GAM based on equation (\ref{Gamma})

%TCIMACRO{%
%\TeXButton{GAM code}{\begin{verbatim}
%C***********************************************************************       
%C*---------------------------------------------------------------------*
%complex Function GAM(d,k,EPS)                                     
%C*---------------------------------------------------------------------*      
%C***********************************************************************
%C*    Function to calculate the self term contribution termed GAMMA    *
%C*      in the IBM write-up. The form for GAM  is taken from the work  *
%C*      of B. T. Draine and J. Goodman, Astrophysical Journal, 405:    *
%C*      685-697, 1993 March 10. Two other self term calculations are   *
%C*      listed here for reference.  In my experience, the Draine and   *
%C*      Goodman formulation is the better of the three.                *
%C* Goedecke and O'Brian: Note the sign change due the time dependance  *
%C*      in IBM being exp(-iwt). This differes from Goedecke and        *
%C*      O'Brien's choice.                                              *
%C*       GAM=(3./(4.*PI))**(2./3.)*(kd)**2 + CI*kd**3/(2.*PI)          *
%C* All The terms in the Goedecke and O'Brien series.  Goedecke and     *
%C*      O'Brien expand the exponential in the self term integral and   *
%C*      throw away most of the trems.  This is the result if you keep  *
%C*      all the terms.                                                 *
%C*       real a                                                        *
%C*       complex temp                                                  *
%C*       a=d*(3./(4.*PI))**(1./3.)                                     *
%C*       temp=CI*k*a                                                   *
%C*       GAM=2.*((1.-CI*k*a)*cexp(temp)-1.)                            * 
%C*  Single value checed 1 Aug 96   formula checked 6 Aug 96            *
%C***********************************************************************
%C**** Variables                                                     ****  
%implicit none
%complex EPS,CI      
%real k,d,kd
%real PI
%parameter (PI=3.141592654,CI=(0.0,1.0)) 
%c       real b1,b2,b3,S
%real b1
%kd=k*d
%C *** Drain and Goodman                                              ***
%c       b1=-1.8915316
%c       b2=0.1648469
%c       b3=-1.7700004
%c       S=1./5.
%c       GAM=(3./(4.*PI))*((b1+EPS*(b2+b3*S))*kd**2+(2.*CI*(kd**3)/3.))
%c      b1=0.0
%b1=(3./(4.*PI))**(2./3.)
%GAM=b1*(kd)**2 + CI*kd**3/(2.*PI)          
%return
%end
%\end{verbatim}}}%
%BeginExpansion
\begin{verbatim}
C***********************************************************************       
C*---------------------------------------------------------------------*
complex Function GAM(d,k,EPS)                                     
C*---------------------------------------------------------------------*      
C***********************************************************************
C*    Function to calculate the self term contribution termed GAMMA    *
C*      in the IBM write-up. The form for GAM  is taken from the work  *
C*      of B. T. Draine and J. Goodman, Astrophysical Journal, 405:    *
C*      685-697, 1993 March 10. Two other self term calculations are   *
C*      listed here for reference.  In my experience, the Draine and   *
C*      Goodman formulation is the better of the three.                *
C* Goedecke and O'Brian: Note the sign change due the time dependance  *
C*      in IBM being exp(-iwt). This differes from Goedecke and        *
C*      O'Brien's choice.                                              *
C*       GAM=(3./(4.*PI))**(2./3.)*(kd)**2 + CI*kd**3/(2.*PI)          *
C* All The terms in the Goedecke and O'Brien series.  Goedecke and     *
C*      O'Brien expand the exponential in the self term integral and   *
C*      throw away most of the trems.  This is the result if you keep  *
C*      all the terms.                                                 *
C*       real a                                                        *
C*       complex temp                                                  *
C*       a=d*(3./(4.*PI))**(1./3.)                                     *
C*       temp=CI*k*a                                                   *
C*       GAM=2.*((1.-CI*k*a)*cexp(temp)-1.)                            * 
C*  Single value checed 1 Aug 96   formula checked 6 Aug 96            *
C***********************************************************************
C**** Variables                                                     ****  
implicit none
complex EPS,CI      
real k,d,kd
real PI
parameter (PI=3.141592654,CI=(0.0,1.0)) 
c       real b1,b2,b3,S
real b1
kd=k*d
C *** Drain and Goodman                                              ***
c       b1=-1.8915316
c       b2=0.1648469
c       b3=-1.7700004
c       S=1./5.
c       GAM=(3./(4.*PI))*((b1+EPS*(b2+b3*S))*kd**2+(2.*CI*(kd**3)/3.))
c      b1=0.0
b1=(3./(4.*PI))**(2./3.)
GAM=b1*(kd)**2 + CI*kd**3/(2.*PI)          
return
end
\end{verbatim}%
%EndExpansion

\bigskip

\bigskip

\subsection{In Order in the Code}

51-78 Declare variables

80 Set inputfile (from vgfin.f), the output file name, and kfile (from
vgfkv.f)

82-83 Declare function types. We will use several functions in this code
that are specific to the code. They are declared here.

87-88 Define some constants

89-98 Interact with the user to get input data

99-116 Open the input file (from vgfin.f) and read in NUSE, Wave, Alpha,
Beta, Gamma, Psi, RAD, Er, EI, TD and the cell positions. Along the way
calculate $\chi $, $\varepsilon ,$ and $m.$

117-135 Make the incoming plane wave and then rotate it into the right
position. This starts with angles $\alpha ,$ $\beta ,$ and $\gamma $ and
converts from degrees to radians. The the function Rot is used to calculate
a rotation matrix. This function calculates the Euler rotation matrix where
the three rotation angles are alpha, a rotation about the z-axis, beta, a
rotation about the new y-axis, and gamma, a rotation about the new z-axis. 
\[
RRR=\left( 
\begin{array}{ccc}
\cos \alpha \cos \beta \cos \gamma -\sin \alpha \sin \gamma  & \cos \alpha
\cos \beta \cos \gamma -\cos \alpha \sin \gamma  & \sin \beta \cos \gamma 
\\ 
-\cos \alpha \cos \beta \sin \gamma -\sin \alpha \cos \gamma  & -\sin \alpha
\cos \beta \sin \gamma +\cos \alpha \cos \gamma  & \sin \beta \sin \gamma 
\\ 
\cos \alpha \sin \beta  & \sin \alpha \sin \beta  & \cos \beta 
\end{array}%
\right) 
\]%
we will start with our input direction being in the $z$-direction%
\[
V_{k}=\left( 
\begin{array}{c}
0 \\ 
0 \\ 
1%
\end{array}%
\right) 
\]%
so our new direction for our incoming wave in the particle frame is 
\[
Khat=RRR\ast V
\]%
where the function MV\ is used to do the multiply. But then we want the $E$%
-vector to be in the $x-y$ plane in the lab frame (can't remember why). So
take 
\[
V_{E}=\left( 
\begin{array}{c}
\cos \psi  \\ 
\sin \psi  \\ 
0%
\end{array}%
\right) 
\]%
then rotate into our particle frame%
\[
E_{Ohat}=RRR\ast V
\]%
and use MV again to do the multiply.

137 Calculate the $W$ factor using function Wcalc%
\[
W\left( \nu \right) =\frac{\chi }{1+\frac{4\pi }{3}\chi } 
\]

138-148 Form the incoming field. This will be of the form 
\[
\mathbf{E}\left( \mathbf{r},t\right) =E_{o}\exp \left( i\mathbf{r}\cdot 
\mathbf{k}\right) \exp (-i\omega t)
\]%
And we want to know the value of the field at every cell. The code makes the 
$\mathbf{r}\cdot \mathbf{k}$ in a loop from 140-142. It has to do this for
each cell. So there is an outside loop from 138 to 148 that loops over cells
and the $r$ components will be different for every cell as we calculated in
vgfin.f. The $\mathbf{r}\cdot \mathbf{k}$ is then multiplied by $i$ in line
143 and then put into an exponent in line 144 and finally the efield
components are assembled in a small loop in lines 145 to 147.

151-154 Now go and get the k-vectors that we built in vgfkv.f. These will be
used to form the outgoing scattered field.

156-242 is the Monty Carlo loop that finds the outgoing scattered field. It
caucluates the $a_{Nj}$ so later we can form  
\[
\tilde{F}_{\beta j}\mathbf{=}\sum_{N=1}^{N_{k}}a_{Nj}\Psi _{\beta N}
\]%
and then 
\[
\mathbf{E}(\mathbf{r})=\frac{\mathbf{F}(\mathbf{r})}{(1+\frac{4\pi }{3}\chi )%
}
\]%
which is what we want as our answer.

The loop starts with finding $\Psi _{\beta N}$%
\[
\Psi _{\beta N}=e^{imk\mathbf{\hat{k}}_{N}\mathbf{\cdot r}_{\beta }}
\]%
\[
\psi =e^{(i\ast k\ast khatN.R(b))}
\]%
and this is done with a call to the function CPHI.

Then it starts to calculate 
\[
ERR=\left\vert \Psi _{\beta N}\right\vert ^{2}
\]%
This will be our optimization test. In line 248 we check to see if $ERR>ERR0$
The $ERR0$ was something we asked the user to input at the start. If $%
ERR<ERR0$ then we are done with one itteration. The original inputs worked.
So the code jumps down to the else clause in line 275 and it calles Ecalc to
calcuate the scattered efield

\end{document}
